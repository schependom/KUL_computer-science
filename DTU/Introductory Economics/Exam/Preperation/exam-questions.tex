\documentclass[dtu]{dtuarticle}
\usepackage{parskip} % use enters instead of indents

\usepackage{amsmath}
\usepackage{amssymb}
\usepackage{bm}
\usepackage{url}
\usepackage{hyperref}
\usepackage{subcaption}
\usepackage{siunitx}
\usepackage[most]{tcolorbox} % most is required for breakable
\usepackage{listings}
\usepackage{color}

\newcommand{\N}{\mathbb{N}}
\newcommand{\Z}{\mathbb{Z}}
\newcommand{\Zn}{\mathbb{Z}_n}
\newcommand{\Zp}{\mathbb{Z}_{\geq 0}}
\newcommand{\Zpp}{\mathbb{Z}_{>0}}
\newcommand{\Q}{\mathbb{Q}}
\newcommand{\R}{\mathbb{R}}
\newcommand{\C}{\mathbb{C}}
\newcommand{\F}{\mathbb{F}}
\newcommand{\E}{\mathbb{E}}
\newcommand{\eqDef}{\overset{\Delta}{=}}
\newcommand{\plusn}{+_n}
\newcommand{\timesn}{\cdot_n}
\newcommand{\id}{\mathrm{id}}
\newcommand{\gen}[1]{\langle #1 \rangle} % subgroup generated by #1
\newcommand{\red}[1]{\color{red}#1\color{black}}
\newcommand{\ord}{\operatorname{ord}}
\newcommand{\lcm}{\operatorname{lcm}}
\newcommand{\im}{\operatorname{im}}
\newcommand{\blank}{$\boxed{\, ? \,}\,\,$}
\newcommand{\PS}{\mathrm{PS}}
\newcommand{\CS}{\mathrm{CS}}
\newcommand{\TS}{\mathrm{TS}}
\DeclareSIUnit\billion{\text{billion}}
\DeclareSIUnit\million{\text{million}}

\newcounter{qnumber}[section]
% \renewcommand{\theqnumber}{\thesubsection.\arabic{qnumber}}
\newenvironment{question}[1][]{
    \def\qpoints{#1}
    \refstepcounter{qnumber}
    \par\medskip\noindent
    % Start a box that spans the full width of the line
    \begin{minipage}{\linewidth} 
        \vspace{0.2cm}
        \textbf{Question \theqnumber:}
}{
    % Check if the argument is empty
    \ifx\qpoints\empty
    \else
        \textbf{(\qpoints\ points)}
    \fi
    % Close the box
    \end{minipage}
    \par\medskip 
}

\newcounter{choice}
\renewcommand\thechoice{\Alph{choice}}
\newcommand\choicelabel{\thechoice.}

\newenvironment{choices}%
  {\list{\choicelabel}%
     {\usecounter{choice}\def\makelabel##1{\hss\llap{##1}}%
       \settowidth{\leftmargin}{W.\hskip\labelsep\hskip 2.5em}%
       \def\choice{%
       \vspace{0.1cm}
         \item
       } % choice
       \labelwidth\leftmargin\advance\labelwidth-\labelsep
       \topsep=1pt
       \partopsep=1pt
     }%
  }%
  {\endlist}

\newenvironment{oneparchoices}%
  {%
    \setcounter{choice}{0}%
    \def\choice{%
      \refstepcounter{choice}%
      \ifnum\value{choice}>1\relax
        \penalty -50\hskip 1em plus 1em\relax
      \fi
      \choicelabel
      \nobreak\enskip
    }% choice
    \def\rightchoice{
        \refstepcounter{choice}
        \ifnum\value{choice}>1\relax
            \penalty -50\hskip 1em plus 1em\relax
        \fi
        \textbf{\choicelabel}
        \nobreak\enskip
    }
    % If we're continuing the paragraph containing the question,
    % then leave a bit of space before the first choice:
    \ifvmode\else\enskip\fi
    \ignorespaces
  }%
  {}

\newtcolorbox{explanation}{
    unbreakable,           % <--- Stops splitting across pages
    colback=white,
    colframe=gray,
    boxrule=0.1mm,           % Removes standard border
    % --- The Header Section ---
    title={EXPLANATION},        % Moves label inside the box logic
    fonttitle=\sffamily\bfseries\footnotesize,
    coltitle=gray,         % Text color
    colbacktitle=white,    % Background white (invisible)
    titlerule=0.2mm,       % This acts as your "toprule"
    bottomtitle=0.1cm,     % Adds the spacing you had in \vspace
    toptitle=0.1cm,
    top=0.2cm, bottom=0.2cm,
    left=0.2cm, right=0.2cm,
    arc=0pt, outer arc=0pt,
    parbox=false
}

\title{Exam Questions}
\subtitle{Introductory Economics}
\author{Vincent Van Schependom}
\course{42009 Introductory Economics}
\address{
	DTU Management \\
	Fall 2025
}
\date{Fall 2025}



\begin{document}

\maketitle

\begin{question}
    Which of the following activities have zero opportunity cost?
    \begin{choices}
        \choice To see a concert with a free ticket obtained from a lucky draw
        \choice To attend college fully on scholarships
        \choice To participate in the demonstration for equality in Nyhavn, Copenhagen
        \choice \textbf{None of the above. All of the events above have opportunity costs.}
    \end{choices}
\end{question}

\begin{question}
    Changes in the price of a good lead to:
    \begin{choices}
        \choice \textbf{Change in the quantity supplied of the good.}
        \choice Changes in supply.
        \choice Changes in demand.
        \choice No effects in quantity supplied or demanded.
    \end{choices}
\end{question}

\begin{explanation}
    \begin{itemize}
        \item[A.] Changing \textit{only}
              the price leads to changes in the \textbf{quantity supplied}. This is graphically
              represented as a movement \textit{along} a given supply curve.
              Hence, \textbf{answer A} is correct.
              $$P_X \text{ changes } \Rightarrow Q_X^S \text{ changes as a function of } P_X$$

        \item[B/C.] `Supply' and demand' are the entire curves/relationships, not single points.
              A change in \textit{supply} refers to a \textbf{shift} of the
              \textit{entire} supply curve. This is caused by changing factors \textit{other} than the price of the
              good itself, such as input prices, technology, the number of firms, or taxes.
              Similary, a change in \textit{demand} refers to a \textbf{shift}
              of the \textit{entire} demand curve caused by factors like income, tastes, or prices of related goods.
              A change in the good's own price leads to a change in the \textit{quantity demanded}
              (movement along the curve), not a change in demand.

        \item[D.] Economic theory (Law of Supply and Law of Demand)
              states that price is the primary determinant of both quantity supplied and quantity demanded;
              therefore, a price change definitely has an effect.
    \end{itemize}
\end{explanation}

\begin{question}
    To solve the traffic congestion problem in Copenhagen, the government is considering policies to reduce private cars on the road.
    Which of the following policy will \textbf{NOT} be helpful in achieving this goal?
    \begin{choices}
        \choice To provide subsidy to the people who ride a bike.
        \choice To implement a higher tax on car purchase.
        \choice To implement a higher tax on gasoline.
        \choice \textbf{To charge a higher price on public transport.}
    \end{choices}
\end{question}

\newpage

\begin{question}
    The quantity demanded of a good decreases by 20\% when its price increase by 2\%.
    Which of the followings best fit this good?
    \begin{choices}
        \choice This good has no close substitutes.
        \choice This is a necessity good.
        \choice \textbf{This is more likely to happen in a short-run.}
        \choice This is an inferior good.
    \end{choices}
\end{question}

\begin{explanation}
    First, let's define the goods based on income elasticity:
    \begin{itemize}
        \item \textbf{Normal good:} Income $\uparrow \implies$ Demand $\uparrow$
        \item \textbf{Inferior good:} Income $\uparrow \implies$ Demand $\downarrow$
    \end{itemize}
    Since the question only provides data on \textit{price} changes, we cannot determine if it is inferior or normal. Thus, option (d) is not the best fit.

    Next, let's calculate the \textbf{Price Elasticity of Demand} ($|E_{Q_X^d, P_X}|$):
    $$|E_{Q_X^d, P_X}| = \left| \frac{\%\Delta Q_X^d}{\%\Delta P_X} \right| = \left| \frac{-20\%}{2\%} \right| = |-10| = 10$$

    Now we classify the good based on elasticity:
    \begin{itemize}
        \item \textbf{Inelastic ($|E| < 1$):} Quantity changes \textit{less} than price. Often associated with \textbf{necessities}, goods with \textbf{no close substitutes}, or the \textbf{short-run}.
        \item \textbf{Elastic ($|E| > 1$):} Quantity changes \textit{more} than price. Often associated with \textbf{luxuries}, goods with \textbf{many substitutes}, or the \textbf{long-run}.
    \end{itemize}

    \textbf{Analysis of the Options:}
    \begin{itemize}
        \item (a) No close substitutes $\implies$ Inelastic.
        \item (b) Necessity $\implies$ Inelastic.
        \item (c) Short-run $\implies$ Inelastic.
    \end{itemize}

    Demand is \textbf{less elastic (more inelastic) in the short-run} because consumers
    need time to adjust their habits and find substitutes. Conversely,
    demand becomes \textbf{more elastic in the long-run}.
\end{explanation}

\newpage

\begin{question}
    Suppose we observe a decrease of the equilibrium price of potato and an increase of the equilibrium quantity.
    Which of the following best fit the observed data?
    \begin{choices}
        \choice An increase in demand with supply unchanged
        \choice A decrease in supply with demand unchanged
        \choice \textbf{An increase in supply with demand unchanged}
        \choice An increase in demand coupled with a decrease in supply
    \end{choices}
\end{question}

\begin{explanation}
    Equilibrium quantity moves in the same direction as the (horizontal) curve shift
    for both supply and demand, but price moves in opposite directions:
    \begin{itemize}
        \item Demand $\uparrow$: $P \uparrow, Q \uparrow$
        \item Supply $\uparrow$: $P \downarrow, Q \uparrow$
    \end{itemize}
    Since $P$ fell and $Q$ rose, this must be a positive supply shift.
\end{explanation}

\begin{question}
    The cross price elasticity between good $X$ and good $Y$ is found to be positive.
    We conclude that good $X$ and good $Y$ are:
    \begin{choices}
        \choice normal goods
        \choice inferior goods
        \choice \textbf{substitutes}
        \choice complements
    \end{choices}
\end{question}

\begin{explanation}
    Cross price elasticity is given by
    $$E_{Q_X^d,P_Y} = \frac{\%\Delta Q_X^d}{\%\Delta P_Y} = \frac{d Q_X^d}{d P_Y} \cdot \frac{P_Y}{Q_X^d} > 0,$$
    which means that if the price of $Y$ goes up, the demand for $X$ goes up.
    This implies consumers are switching from $Y$ to $X$,
    making them substitutes.
\end{explanation}

\begin{question}
    The CEO of a large restaurant chain said, ``For each 1 percent price increase, we lose 5 percent of our diners.''
    We can conclude that:
    \begin{choices}
        \choice demand is price inelastic.
        \choice \textbf{a price increase will decrease total revenue.}
        \choice the price elasticity is $-0.5$.
        \choice the demand curve is horizontal.
    \end{choices}
\end{question}

\begin{explanation}
    The elasticity is $E_{Q_X^d, P_X} = (-5\%)/(+1\%) = -5$. Since $|E_{Q_X^d, P_X}| > 1$, demand is elastic.
    When demand is elastic, price increase leads to a decrease in total revenue.

    The demand curve is horizontal if the demand is perfectly elastic, i.e. $$E_{Q_X^d, P_X} = -\infty \iff \frac{d Q_X^d}{d P_X} = 0$$
\end{explanation}

\begin{question}
    Suppose the (inverse) demand for a product is $P=40-Q$ and (inverse) supply of the product is $P=4+2Q$. The equilibrium quantity, price, and consumer surplus ($\CS$) would be:
    \begin{choices}
        \choice $\boxed{Q=12, P=28, \CS=72}$
        \choice $Q=8, P=14, \CS=36$
        \choice $Q=12, P=28, \CS=36$
        \choice $Q=8, P=14, \CS=72$
    \end{choices}
\end{question}

\begin{explanation}
    Set Demand = Supply:
    $$40 - Q = 4 + 2Q \implies 36 = 3Q \implies Q = 12$$
    Substitute $Q$ back to find $P$:
    $$P = 40 - 12 = 28$$
    Consumer Surplus is the area of the triangle below the demand intercept ($P=40$) and above the market price ($P=28$):
    $$\CS = 0.5 \times \text{Base} \times \text{Height} = 0.5 \times 12 \times (40-28) = 0.5 \times 12 \times 12 = 72$$
\end{explanation}

\begin{question}
    A price ceiling is
    \begin{choices}
        \choice the minimum legal price that can be charged in a market.
        \choice \textbf{the maximum legal price that can be charged in a market.}
        \choice higher than the initial equilibrium price.
        \choice equal to the initial equilibrium price.
    \end{choices}
\end{question}

\begin{question}
    Suppose the supply curve of a product is $Q = -10 + 10P$. The producer surplus will increase by the amount of
    \blank if the price rises from 2 DKK to 3 DKK per unit.
    \begin{choices}
        \choice 5 DKK
        \choice 10 DKK
        \choice \textbf{15 DKK}
        \choice 20 DKK
    \end{choices}
\end{question}

\begin{explanation}
    First note that inverse supply curve is $P(Q) = \frac{1}{10}Q + 1 \implies \red{P(0) = 1}$.
    Producer surplus is the area above the supply curve and below the market price from $Q=0$ to $Q=Q^*$,
    where $Q^*$ is the quantity supplied at the market price.
    \begin{align*}
        Q^*_1      & = -10 + 10 \cdot 2 = 10                         \\
        Q^*_2      & = -10 + 10 \cdot 3 = 20                         \\
        \PS_1      & = \frac{1}{2} \cdot 10 \cdot (2 - \red{1}) = 5  \\
        \PS_2      & = \frac{1}{2} \cdot 20 \cdot (3 - \red{1}) = 20 \\
        \Delta \PS & = \PS_2 - \PS_1 = 20 - 5 = 15
    \end{align*}
\end{explanation}

\begin{question}
    Suppose the production function is given by $Q=\min\{3K,4L\}$. What is name of this production function form and what is the average product of labor $(AP_{L})$ when 15 units of capital and 10 units of labor are employed?
    \begin{choices}
        \choice Cobb-Douglas, $AP_{L}=4$
        \choice \textbf{Leontief, $AP_{L}=4$}
        \choice Cobb-Douglas, $AP_{L}=3$
        \choice Leontief, $AP_{L}=3$
    \end{choices}
\end{question}

\begin{explanation}
    The form $Q=\min\{aK,bL\}$ is a Leontief (fixed proportion) function.\\
    Substitute $K=15, L=10, a=3, b=4$:
    $$Q = \min\{3 \cdot 15, 4 \cdot 10\} = \min\{45, 40\} = 40$$
    Average Product of Labor:
    $$AP_L = \frac{TP}{L} = \frac{40}{10} = 4$$
\end{explanation}

\begin{question}
    Suppose that a firm produces output according to the production function $Q = K^{1/4}L^{3/4}$, what is the marginal product of
    labor when 1 unit of capital and 16 units of labor are employed?
    \begin{choices}
        \choice \textbf{3/8}
        \choice 5/8
        \choice 3/4
        \choice 5/4
    \end{choices}
\end{question}

\begin{explanation}
    For this Cobb-Douglas production function $Q = K^{\alpha}L^{\beta}$, the marginal product of labour is:
    $$MP_L = \beta K^{\alpha}L^{\beta-1}$$
    When $K=1$ and $L=16$:
    $$MP_L = \frac{3}{4} \cdot 1^{1/4} \cdot 16^{-1/4} = \frac{3}{4} \cdot 1 \cdot \frac{1}{2} = \frac{3}{8}$$
\end{explanation}

\begin{question}
    Suppose the production function is $Q=3K + 4L$. The marginal rate of technical substitution is:
    \begin{choices}
        \choice 4/3
        \choice 2/3
        \choice 8/3
        \choice 5/6
    \end{choices}
\end{question}

\begin{explanation}
    For a linear production function $Q = aK + bL$, the marginal products are \textit{constant}:
    $$MP_K = a = 3 \quad \land \quad MP_L = b = 4 \quad \implies \quad MRTS_{LK} = \frac{MP_L}{MP_K} = \frac{4}{3}$$
\end{explanation}

\begin{question}
    A firm uses labor ($L$) and capital ($K$) as inputs to produce.
    If the price of inputs are $w=60$ DKK, $r = 200$ DKK,
    and marginal products are $MP_{L}=30, MP_{K}=100$, the firm:
    \begin{choices}
        \choice \textbf{is cost minimizing.}
        \choice should use less $L$ and more $K$ to cost minimize.
        \choice should use less $K$ and more $L$ to cost minimize.
        \choice is profit maximizing but not cost minimizing.
    \end{choices}
\end{question}

\begin{explanation}
    Cost minimization is producing at the \textit{lowest possible price for a given level of output} $Q_I$:
    $$\text{Cost for } Q_I \text{ minimized} \iff \frac{MP_L}{w} = \frac{MP_K}{r} \iff MRTS_{KL} = \frac{w}{r}$$
    In this case, we calculate that
    $$\frac{MP_L}{w} = \frac{30}{60} = 0.5 = \frac{100}{200} = \frac{MP_K}{r}$$
    Since the ratios are equal, the firm is indeed cost minimizing.
    In the case of inequal ratios:
    \begin{align*}
        \frac{MP_L}{w} > \frac{MP_K}{r} & \implies \text{use more } L \text{ and less } K \\
        \frac{MP_L}{w} < \frac{MP_K}{r} & \implies \text{use less } L \text{ and more } K
    \end{align*}
\end{explanation}

\begin{question}
    Regarding isoquants and isocosts, which of the followings is NOT correct?
    \begin{choices}
        \choice An isoquant defines the combinations of inputs that yield the producer the same level of output.
        \choice An isocost line defines the combinations of inputs that yield the producer the same cost.
        \choice \textbf{An isoquant should never intersect with an isocost line.}
        \choice The producer is cost minimizing at the point of tangency between an isoquant and an isocost line.
    \end{choices}
\end{question}

\begin{question}
    Regarding the average and marginal costs, which of the following is NOT correct?
    \begin{choices}
        \choice Average total cost increases when marginal cost curve is above the average total cost curve.
        \choice The marginal cost curve intersects the average total cost curve at the minimum point of average total cost curve.
        \choice The marginal cost curve intersects the average variable cost curve at the minimum point of average variable cost curve.
        \choice \textbf{Marginal cost decreases when average fixed cost decreases.}
    \end{choices}
\end{question}

\begin{explanation}
    Average Fixed Cost (AFC) \textit{always} decreases as output increases.
    Marginal Cost (MC), however, is typically U-shaped (it eventually rises due to diminishing returns).
    There is no rule stating MC must fall just because AFC is falling.
    The other three statements are fundamental laws of cost curves.
\end{explanation}

\begin{question}
    Economies of scale exist when
    \begin{choices}
        \choice \textbf{average total costs decline as output increases.}
        \choice average total costs increase as output increases.
        \choice average total costs remains constant as output increases.
        \choice average fixed costs decline as output increases.
    \end{choices}
\end{question}

\begin{question}
    Generally, an increase in the number of vegetarians will cause the demand curve for meat to
    \begin{choices}
        \choice shift rightward.
        \choice \textbf{shift leftward.}
        \choice become flatter.
        \choice become steeper.
    \end{choices}
\end{question}

\begin{question}
    Which of the following would cause the current supply curve of iPhone to shift rightward?
    \begin{choices}
        \choice an economic boom, which increases the amount that people are willing to spend on personal electronics
        \choice a decrease in the price of songs on Apple music
        \choice \textbf{the producer's expectation that the future price of iPhone will decrease}
        \choice an increase in the wages of labor in iPhone manufacturers
    \end{choices}
\end{question}

\begin{explanation}
    \begin{itemize}
        \item A and B affect \textit{demand}, not supply.
              As WTP's increase, demand shifts right.
              As price decreases, demand shifts left.
        \item D (higher wages) increases costs, shifting supply \textit{left}.
        \item C suggests sellers want to sell now before prices drop, increasing current supply (shifting \textbf{right}).
    \end{itemize}
\end{explanation}

\begin{question}
    Regarding accounting and economic profits/costs, which of the following is NOT correct?
    \begin{choices}
        \choice Accounting profits generally overstate economic profits.
        \choice Accounting profits do not take opportunity cost into account.
        \choice Economic costs include not only the accounting costs but also the opportunity costs of the resources used in production.
        \choice \textbf{Managers should only care about accounting profits.}
    \end{choices}
\end{question}

\begin{explanation}
    A and B are correct, because accounting profit does not take opportunity cost into account.
    \begin{align*}
        \text{Accounting profit} & = \text{Revenue} - \text{Explicit costs}                           \\
        \text{Economic profit}   & = \text{Revenue} - \text{Opportunity costs}                        \\
                                 & = \text{Revenue} - (\text{Explicit costs} + \text{Implicit costs}) \\
                                 & = \text{Accounting profit} - \text{Implicit costs}
    \end{align*}
    C is correct, because $\text{Economic cost} = \text{Accounting cost} + \text{Implicit cost}$
\end{explanation}

\begin{question}
    A firm in a competitive market sells its product at a price of 60 DKK and its cost function is $C(Q)=20+5Q^{2}.$ The maximum profits for the firm would be:
    \begin{choices}
        \choice \textbf{160 DKK}
        \choice 100 DKK
        \choice 360 DKK
        \choice 200 DKK
    \end{choices}
\end{question}

\begin{explanation}
    Profit maximization in perfect competition occurs where $P = MC$.
    $$MC = \frac{dC}{dQ} = 10Q$$
    So we calculate the profit-maximizing quantity $Q^*$:
    \begin{align*}
        P = MC(Q^*) \iff 60 = 10Q^* \iff Q^* = 6
    \end{align*}
    Now calculate profit ($\pi = TR - TC$):
    \begin{align*}
        TR  & = P \cdot Q^* &  & = 60 \cdot 6 = 360               \\
        TC  & = C(Q^*)      &  & = 20 + 5(6)^2 = 20 + 5(36) = 200 \\
        \pi & = TR - TC     &  & = 360 - 200 = 160
    \end{align*}
\end{explanation}

\begin{question}
    Suppose you are a supervisor of PhD student, which of the following is NOT a solution to the principle-agent problem of supervisor-student?
    \begin{choices}
        \choice To give bonus for publication in journals/conferences
        \choice To give bonus for project reports
        \choice Spot checks at the office of PhD student
        \choice \textbf{Fixed salary regardless of performance}
    \end{choices}
\end{question}

\begin{question}
    Regarding fixed costs and sunk costs, which of the following is NOT correct?
    \begin{choices}
        \choice Sunk costs are those costs that are forever lost after they have been paid.
        \choice Fixed costs do not vary with output.
        \choice A lost ticket before a movie starts is an example of sunk cost.
        \choice \textbf{Sunk costs could be part of the marginal costs.}
    \end{choices}
\end{question}

\begin{explanation}
    Marginal cost looks forward (the cost of the \textit{next} unit). Sunk costs are in the past.
    Therefore, sunk costs can never be part of marginal costs.
\end{explanation}

\begin{question}
    The sources of monopoly power for a monopoly could be:
    \begin{choices}
        \choice economies of scale.
        \choice economies of scope.
        \choice patents.
        \choice \textbf{all of the above.}
    \end{choices}
\end{question}

\begin{explanation}
    Economies of scale: $ATC \downarrow$ as $Q \uparrow$.\\
    Economies of scope: $C(Q_1,0) + C(0,Q_2) > C(Q_1,Q_2)$.
\end{explanation}

\begin{question}
    The long-run equilibrium of a perfectly competitive market is characterized by:
    \begin{choices}
        \choice $P >$ minimum of ATC
        \choice $P <$ AVC
        \choice \textbf{P = MC = minimum of ATC (average total cost)}
        \choice $P > MR$
    \end{choices}
\end{question}

\begin{explanation}
    In the long-run competitive equilibrium, the price equals the minimum of average total cost:
    $$P=MC \quad \text{and} \quad P^e = \min ATC(Q)$$
    with $Q^* = \text{argmin } ATC(Q)$
\end{explanation}

\begin{question}
    A perfectly competitive firm will shut down (stop producing) when:
    \begin{choices}
        \choice market price is lower than the average total cost (ATC).
        \choice total revenue is less than the total cost.
        \choice \textbf{market price is lower than the average variable cost (AVC).}
        \choice fixed cost is too high.
    \end{choices}
\end{question}

\begin{explanation}
    A firm should shut down in the short run if the price is less
    than the average variable cost at the profit-maximizing quantity $Q^*$ for which $MR = MC$:
    $$
        \text{Shut down} \iff P < AVC(Q^*)
    $$
\end{explanation}

\begin{question}
    Which of the following statements about a monopoly is NOT correct?
    \begin{choices}
        \choice A monopoly does not have a supply curve.
        \choice The demand curve of a monopoly is the market demand curve.
        \choice \textbf{A monopoly has no market power.}
        \choice A monopoly produces at $MR=MC$.
    \end{choices}
\end{question}

\begin{explanation}
    A profit-maximizing monopolist produces $Q^M$ for which $MR(Q^M) = MC(Q^M)$.
    Given this optimal level of output $Q^M$, the monopolist sets the price
    according to the demand curve: $P^M = P(Q^M)$.

    A monopolist's market power implies that $P > MR = MC$. This means that
    there is no supply curve, because there is no one-to-one relationship between
    price and quantity supplied.

    The same applies for firms with market power
    in a monopolistic competition or oligopoly.
\end{explanation}

\begin{question}
    A monopoly faces a demand curve described by $P=90-3Q$ and has a total cost of $C(Q)=5+10Q+Q^{2}.$ The profit-maximizing price for the monopoly is:
    \begin{choices}
        \choice \textbf{60}
        \choice 10
        \choice 30
        \choice 20
    \end{choices}
\end{question}

\begin{explanation}
    \begin{enumerate}
        \item Derive MR: $P=90-\red{3}Q \implies MR = 90 - \red{6}Q$.
        \item Derive MC: $C(Q)=5+10Q+Q^2 \implies MC(Q) = 10 + 2Q$.
        \item Set $MR(Q) = MC(Q)$:
              $$90 - 6Q = 10 + 2Q \implies 80 = 8Q \implies Q = 10$$
        \item Find Price:
              $$P = 90 - 3 \cdot 10 = 60$$
    \end{enumerate}
\end{explanation}

\begin{question}
    Which of the following is NOT an example of negative externalities?
    \begin{choices}
        \choice Air pollution from a factory
        \choice The neighbor's barking dog
        \choice Health risk to others from second-hand smoke
        \choice \textbf{Being vaccinated against Covid-19 protects not only you, but also the people around you.}
    \end{choices}
\end{question}

\begin{question}
    Consider a monopoly where the inverse demand for its product is given by $P=50-2Q$.
    Total costs for this monopolist is $C(Q)=100+2Q+Q^{2}$. At the profit-maximizing combination of output and price, deadweight loss is:
    \begin{choices}
        \choice \textbf{32}
        \choice 64
        \choice 128
        \choice cannot be determined with the given information.
    \end{choices}
\end{question}

\begin{explanation}
    The deadweight loss is the total surplus (TS) lost by society
    because the monopolist produces less than the socially optimal output $Q^C$.

    A profit-maximizing monopolist produces $Q^M$ for which $MR(Q^M) = MC(Q^M)$.
    We determine the profit-maximizing quantity $Q^M$:
    \begin{align*}
        MR(Q^M)   = MC(Q^M) & \iff 50 - 4Q^M = 2 + 2Q^M    \\
                            & \iff 48               = 6Q^M \\
                            & \iff Q^M             = 8
    \end{align*}
    To set the price, a monopolist uses the demand curve: $P^M = P(Q^M) = 50 - 2 \cdot 8 = 34$.\\
    The social optimum is where $P = MC$:
    \begin{align*}
        P(Q^C) = MC(Q^C) & \iff 50 - 2Q^C = 2 + 2Q^C \\
                         & \iff 48 = 4Q^C            \\
                         & \iff Q^C = 12
    \end{align*}

    The DWL is the area of triangle between $Q^M$ and $Q^C$.
    \begin{align*}
        DWL & = \frac{1}{2} \cdot (Q^C - Q^M) \cdot [P(Q^M) - MC(Q^M)] \\
            & = \frac{1}{2} \cdot (12 - 8) \cdot (34 - 18)             \\
            & = 32
    \end{align*}
\end{explanation}

\begin{question}
    Which of the following is NOT a transaction cost associated with using inputs?
    \begin{choices}
        \choice Time spent negotiating labor contracts with union workers
        \choice Opportunity costs of negotiating the price of renting machines.
        \choice \textbf{Wages paid to labor}
        \choice Costs of searching for new supplier of machines
    \end{choices}
\end{question}

\begin{explanation}
    Transaction costs are the costs of making an economic exchange (search, bargaining, enforcement).
    The wage itself is the \textit{price} of the input, not a transaction cost.
\end{explanation}

\begin{question}
    In a free market in which an equilibrium price and quantity prevails:
    \begin{choices}
        \choice consumer surplus is less than producer surplus.
        \choice consumer surplus is greater than producer surplus.
        \choice consumer surplus is the same as producer surplus.
        \choice \textbf{the sum of consumer surplus and producer surplus are maximized.}
    \end{choices}
\end{question}

\begin{explanation}
    In a free market, $TS = CS + PS$ is maximized at the equilibrium price and quantity:
    \begin{align*}
        P(Q^E) = MC(Q^E)
    \end{align*}
    We can't know if $CS$ or $PS$ is larger without the demand curve.
\end{explanation}

\begin{question}
    You are a division manager at Toyota. If your marketing department estimates that
    the semiannual demand for the Highlander is $Q=\num{150000}-1.5P$, what price should you
    charge in order to maximize revenues from sales of the Highlander?
    \begin{choices}
        \choice \textbf{50,000}
        \choice 30,000
        \choice 100,000
        \choice 60,000
    \end{choices}
\end{question}

\begin{explanation}
    Revenue = $P \cdot Q$ is maximized when $MR = 0$.
    $$R(Q) = \num{150000}P - \num{1.5}P^2 \implies MR(Q) = \num{150000} - \num{3}P$$
    Set $MR(Q) = 0$:
    $$\num{150000} - \num{3}P = 0 \implies P = \num{50000}$$
\end{explanation}

\begin{question}
    In a competitive market, the market demand is $Q^{d}=70-3P$ and the market supply is $Q^{S}=6P$. A price ceiling of 4 will result in a
    \begin{choices}
        \choice shortage of 24 units.
        \choice \textbf{shortage of 34 units.}
        \choice surplus of 24 units.
        \choice surplus of 34 units.
    \end{choices}
\end{question}

\begin{explanation}
    % Calculate inverse demand and supply:
    % \begin{align*}
    %     Q^d(P) = 70 - 3P & \implies P^d(Q_d) = \frac{70 - Q_d}{3} \\
    %     Q^s(P) = 6P      & \implies P^s(Q_s) = \frac{Q_s}{6}
    % \end{align*}
    At the price ceiling of 4, the quantity demanded is:
    \begin{align*}
        Q^d(4) = 70 - 3 \cdot 4 = 58
    \end{align*}
    The quantity supplied is:
    \begin{align*}
        Q^s(4) = 6 \cdot 4 = 24
    \end{align*}
    The shortage is:
    \begin{align*}
        \text{Shortage} = Q^d(4) - Q^s(4) = 58 - 24 = 34
    \end{align*}
    We don't need the equilibrium price and quantity to answer this question.
\end{explanation}

\begin{question}
    Andy, a college student, loves eating burger. As a college student with no income,
    he is used to eating at McDonald's. After graduation, Andy gets a job. As such,
    his income is now \num{200000} DKK a year. He ends up eating burger at Sporvejen.
    In economic terms, the burger at McDonald is a(n) \blank while the burger at Sporvejen is a(n) \blank.
    \begin{choices}
        \choice normal good; normal good.
        \choice inferior good; inferior good.
        \choice normal good; inferior good.
        \choice \textbf{inferior good; normal good.}
    \end{choices}
\end{question}

\begin{question}
    Consider a monopoly where the inverse demand for its product is given by $P=100-3Q$. Base on this information, the marginal revenue function is:
    \begin{choices}
        \choice $MR(Q)=200-1.5Q$
        \choice $\boxed{MR(Q)=100-6Q}$
        \choice $MR(Q)=100-1.5Q$
        \choice $MR(Q)=200-6Q$
    \end{choices}
\end{question}

\begin{explanation}
    For a linear demand curve $P = a - bQ$, the Marginal Revenue curve has the same intercept ($a$)
    but twice the slope ($2b$), since $MR = (P \cdot Q)' = (aQ - bQ^2)' = a - 2bQ$.
    $$P = 100 - 3Q \implies MR = 100 - \red{2} \cdot 3 Q = 100 - 6Q$$
\end{explanation}

\begin{question}
    The recipe that defines the maximum amount of output that can be produced with K units of capital and L units of labor is the
    \begin{choices}
        \choice \textbf{production function.}
        \choice cost function.
        \choice marginal product.
        \choice average product.
    \end{choices}
\end{question}

\begin{question}
    Suppose the demand for a product is $\ln Q^{d}=20-2 \ln P$, then this product is
    \begin{choices}
        \choice \textbf{elastic}
        \choice inelastic
        \choice unitary elastic
        \choice cannot be determined.
    \end{choices}
\end{question}

\begin{explanation}
    For a log-linear demand function
    $$\ln Q_X^d = \beta_0 + \beta_X \ln P_X + \beta_Y \ln P_Y + \beta_M \ln M + \beta_H \ln H$$
    We have $$\text{Own el. } E_{Q_X^d, P_X} = \beta_X, \qquad \text{Cross el. } E_{Q_X^d, P_Y} = \beta_Y, \qquad \text{Income el. } E_{Q_X^d, M} = \beta_M$$
    So in this case, the product's own demand is elastic, since $|\beta_X| = 2 > 1$.
\end{explanation}

\begin{question}
    When a demand curve is linear,
    \begin{choices}
        \choice the elasticity is the same as the slope of the demand curve.
        \choice \textbf{demand is elastic at high prices.}
        \choice demand is unitary elastic at low prices.
        \choice the elasticity is constant at all prices.
    \end{choices}
\end{question}

\begin{explanation}
    For a linear demand curve $Q_X^d = a + bP_X$ with $\red{b < 0}$, since $\red{P^X \uparrow \implies Q_X^d \downarrow}$ by law of demand.
    Note that elasticity \textit{varies} along a linear demand curve:
    $$E_{Q_X^d, P_X} = \frac{dQ_X^d}{dP_X} \cdot \frac{P_X}{Q_X^d} = b \cdot \frac{P_X}{a + bP_X}$$

    We see that $|E_{Q_X^d, P_X}| > 1$ at high prices, because as $P_X$ rises, $Q_X^d$
    \textbf{decreases} (the denominator gets smaller) while the numerator gets larger.
    This causes the ratio $P_X/Q_X^d$ to increase (see Figure \ref{fig:elasticity}).

    Also note:
    \begin{itemize}
        \item Horizontal curve $\rightarrow$ Elasticity = $-\infty$ $\rightarrow$ perfectly elastic
        \item Vertical curve $\rightarrow$ Elasticity = $0$ $\rightarrow$ perfectly inelastic
    \end{itemize}
\end{explanation}

\begin{figure}
    \centering
    \includegraphics[width=0.99\textwidth]{images/elasticity.jpg}
    \caption{Elasticity of a linear demand curve. Note that $Q(x) \equiv Q^d_X(P_X)$.}
    \label{fig:elasticity}
\end{figure}

\begin{question}
    If a firm's production function is Leontief and the wage rates goes up, the
    \begin{choices}
        \choice \textbf{cost minimizing combination of capital and labor does not change.}
        \choice firm would use more labor to minimize the cost for a given output
        \choice firm would use more capital to minimize the cost for a given output
        \choice firm would use less labor to minimize the cost for a given output
    \end{choices}
\end{question}

\begin{explanation}
    A Leontief production function has the form
    $$Q = \min(aK, bL)$$
    If one input is fixed, the other is \textit{adjusted to maintain the desired level of output}.

    For example, $Q = \min(2K, 3L)$ means that 2 units of capital and 3 units of
    labor are needed to produce 1 unit of output. If either input is fixed, the other
    input must be adjusted to maintain the desired level of output.

    However, Leontief functions represent perfect complements (L-shaped isoquants).
    The cost-minimizing point is always at the ``vertex'' or corner of the L-shape, where
    $$aK = bL$$
    Since a fixed ratio of inputs is required, regardless of price,
    the firm cannot substitute capital for labor. Therefore, the optimal combination of
    $K$ and $L$ remains exactly the same, even though the total cost to the firm has increased.
\end{explanation}

\begin{question}
    Suppose that you pay zero fee for an additional year of study in DTU.
    However, you could use this year to work in a company instead and get an annual salary
    as high as \num{300000} DKK.
    What is the opportunity cost of attending DTU for one year?
    \begin{choices}
        \choice \num{0} DKK
        \choice \textbf{300,000 DKK}
        \choice \num{150000} DKK
        \choice \num{450000} DKK
    \end{choices}
\end{question}

\begin{question}
    The Lyngby municipality wants to increase the number of households who use electric heat
    pumps for heating. Which of the following policies will NOT be helpful in achieving this
    goal?
    \begin{choices}
        \choice Providing subsidies to households who buy electric heat pumps.
        \choice Increasing the tax rate on natural gas
        \choice Imposing a tax on the sale of gas boilers.
        \choice \textbf{Providing subsidies for the purchase of gas boiler.}
    \end{choices}
\end{question}

\begin{question}
    The quantity demanded of a good would decrease by 3\% when the price of another good
    increases by 5\%. Which of the following could best fit the two goods?
    \begin{choices}
        \choice \textbf{The two goods are complements.}
        \choice The two goods are inferior goods.
        \choice The two goods are substitutes.
        \choice None of the above is correct.
    \end{choices}
\end{question}

\begin{question}
    If the (own) price elasticity of demand for a good is ZERO, then
    \begin{choices}
        \choice the good have perfect substitute.
        \choice the demand curve is horizontal.
        \choice the demand will be infinite.
        \choice \textbf{demand is perfectly inelastic.}
    \end{choices}
\end{question}

\begin{explanation}
    The own price elasticity of demand is given by
    $$E_{Q_X^d, P_X} = \frac{\%\Delta Q_X^d}{\%\Delta P_X} = \frac{d Q_X^d}{d P_X} \cdot \frac{P_X}{Q_X^d}$$
    If $E_{Q_X^d, P_X} = 0$, then
    $$d Q_X^d/d P_X = 0 \implies \text{demand curve is perfectly \textbf{vertical}} \implies \text{demand is perfectly \textbf{inelastic}}$$
\end{explanation}

\begin{question}
    Which of the following is NOT a \textbf{supply} shifter of the Korean fried chicken (Padak) by YATAI
    K-FOOD at DTU?
    \begin{choices}
        \choice Price of raw chicken
        \choice Tax on chicken
        \choice \textbf{Average income level of DTU students} (\textit{demand} shifter!)
        \choice Price of labor
    \end{choices}
\end{question}

\begin{explanation}
    Examples of \textbf{\textit{supply} shifters} are:
    \begin{itemize}
        \item Change in price of input (wage $w$ and rental price of capital $r$)
        \item Change in technology
        \item Change in government regulation (tax, subsidy)
        \item Number of firms
    \end{itemize}
    Examples of \textbf{\textit{demand} shifters} are:
    \begin{itemize}
        \item Income (normal, inferior goods)
        \item Price of related goods (substitute, complement goods)
        \item Advertisement
        \item Consumer taste
        \item Population (age distribution, demographics)
        \item Expectations about future prices and income
    \end{itemize}
\end{explanation}

\begin{question}
    In a competitive market, the market demand is $Q^d = 80 - 4P$ and the market supply is $Q^s = 20 + 6P$.
    A price floor of 10 will result in a
    \begin{choices}
        \choice shortage of 50 units.
        \choice \textbf{shortage of 40 units.}
        \choice surplus of 40 units.
        \choice surplus of 50 units.
    \end{choices}
\end{question}

\begin{explanation}
    \begin{align*}
        Q^d(10) & = 80 - 4 \cdot 10 = 40                                        \\
        Q^s(10) & = 20 + 6 \cdot 10 = 80                                        \\
        Q^d(10) & < Q^s(10) \implies \text{shortage of } Q^s(10) - Q^d(10) = 40
    \end{align*}
\end{explanation}

\begin{question}
    Suppose the production function is $Q=6K + 8L$. The marginal rate of technical substitution is:
    \begin{choices}
        \choice 5/6
        \choice 8/3
        \choice 2/3
        \choice \textbf{4/3}
    \end{choices}
\end{question}

\begin{explanation}
    For a linear production function $Q = aK + bL$, the marginal products are \textit{constant}:
    $$MP_K = a = 6 \quad \land \quad MP_L = b = 8 \quad \implies \quad MRTS_{LK} = \frac{MP_L}{MP_K} = \frac{8}{6} = \frac{4}{3}$$
\end{explanation}

\begin{question}
    Suppose the production function is given by $Q = K^{1/2}L^{1/2}$. What is the name of this
    production function form and what is the average product of labor (APL) when 16 units of
    capital and 16 units of labor are employed?
    \begin{choices}
        \choice Leontief, APL=2
        \choice \textbf{Cobb-Douglas, APL=1}
        \choice Cobb-Douglas, APL=2
        \choice Leontief, APL=1
    \end{choices}
\end{question}

\begin{explanation}
    The production function $Q = K^\alpha L^\beta$ is a Cobb-Douglas production function.
    The average product of labor is
    $$APL = K^\alpha L^{\beta-1}=16^{1/2}16^{-1/2}=1$$
\end{explanation}

\begin{question}
    Which of the following is an example of economies of scope?
    \begin{choices}
        \choice The long-run average cost of aircraft production decreases as output increases.
        \choice \textbf{Apple produces both tablets (iPad) and cellphone (iPhone).}
        \choice The demand of mRNA vaccine decreased after the pandemic.
        \choice The demand for PCs decreased due to the popularity of smart phones.
    \end{choices}
\end{question}

\begin{explanation}
    Economies of scope occur when the cost of producing two goods decreases as the scale of production increases.
    $$C(Q_1,Q_2) < C(Q_1,0)+C(0,Q_2)$$
\end{explanation}

\begin{question}
    Which of the following markets is close to a perfectly competitive market?
    \begin{choices}
        \choice \textbf{agricultural market}
        \choice soft drinks market
        \choice the public transport sector in Denmark
        \choice the PC operating system market
    \end{choices}
\end{question}

\begin{explanation}
    The agricultural market is close to a perfectly competitive market, since farmers are price takers.\\
    Soft drinks market is close to a monopolistic competition market.\\
    The public transport sector in Denmark is close to a monopolistic market.\\
    The PC operating system market is close to a oligopolistic market (a few large firms).
\end{explanation}

\begin{question}
    Suppose that your willingness to pay for a movie on this Friday is 250 DKK, and you have
    bought the ticket for 200 DKK. However, you find you lost your ticket before the movie
    starts. Then the price of your ticket, 200 DKK, is called:
    \begin{choices}
        \choice \textbf{sunk cost}
        \choice fixed cost.
        \choice variable cost.
        \choice consumer surplus.
    \end{choices}
\end{question}

\begin{question}
    A monopoly faces a demand curve described by $P = 70 - 2Q$ and has a total cost of $C(Q) = 8 +
        10Q + Q^2$. The profit-maximizing price for the monopoly is:
    \begin{choices}
        \choice 25
        \choice 100
        \choice \textbf{50}
        \choice 60
    \end{choices}
\end{question}

\begin{explanation}
    A profit-maximizing monopolist produces $Q^M$ for which $MR(Q^M) = MC(Q^M)$.
    We determine the profit-maximizing quantity $Q^M$:
    \begin{align*}
        MR(Q^M)   = MC(Q^M) & \iff 70 - 4Q^M = 10 + 2Q^M   \\
                            & \iff 60               = 6Q^M \\
                            & \iff Q^M             = 10
    \end{align*}
    To set the price, a monopolist uses the demand curve: $P^M = P(Q^M) = 70 - 2 \cdot 10 = 50$.
\end{explanation}

\begin{question}
    Suppose you are the CEO of a company, which of the following is NOT a solution to the
    manager-worker principle-agent problem?
    \begin{choices}
        \choice \textbf{Fixed salary regardless of the worker's performance}
        \choice To give the worker a bonus based on the company's profit
        \choice To give the worker some share of the company
        \choice Spot checks at the company
    \end{choices}
\end{question}

\begin{question}
    Which of the following is NOT an example of positive externalities?
    \begin{choices}
        \choice \textbf{The noise due to the road construction in DTU}
        \choice Research and development create knowledge others can use.
        \choice Being vaccinated against contagious diseases protects not only you, but also other
        people around you.
        \choice People going to college raise the population's education level, which reduces crime
        and improves government.
    \end{choices}
\end{question}

\begin{question}
    A firm in a perfectly competitive market sells its product at a price of 50 DKK and its cost
    function is $C(Q) = 5 + 10Q + 4Q^2$. The maximum profits for the firm would be:
    \begin{choices}
        \choice \textbf{95 DKK}
        \choice 105 DKK
        \choice 55 DKK
        \choice 150 DKK
    \end{choices}
\end{question}

\begin{explanation}
    \begin{align*}
        TR  & = P \cdot Q = 50Q &  & \implies MR = 50                            \\
        TC  & = C(Q)            &  & \implies MC = 10 + 8Q                       \\
        \pi & = TR - TC         &  & \implies \frac{d\pi(Q)}{dQ} = MR(Q) - MC(Q)
    \end{align*}
    We find that
    $$\frac{d\pi(Q^*)}{dQ^*} = 0 \iff 50 - (10 + 8Q^*) = 0 \iff Q^* = 5$$
    The maximum profit is
    $$\pi(Q^*) = 50 \cdot 5 - (5 + 10 \cdot 5 + 4 \cdot 5^2) = 250 - (5 + 50 + 100) = 95$$
\end{explanation}

\begin{question}
    Which of the following is NOT the condition for a perfectly competitive firm to shut down
    (stop producing)?
    \begin{choices}
        \choice \textbf{The revenue can help recover part of the fixed cost.}
        \choice Fixed cost is too high.
        \choice The market price is lower than the average variable cost (AVC).
        \choice The total revenue is smaller than the total variable cost.
    \end{choices}
\end{question}

\begin{explanation}
    C and D are true because
    $$\text{shutdown in short run} \iff P < AVC(Q^*) \iff TR < VC(Q^*)$$
    Fixed costs are sunk in the short run and do \textbf{not} affect the decision to shut down.
\end{explanation}

\begin{question}
    Regarding the supply curve of a perfectly competitive firm, which of the following is correct?
    \begin{choices}
        \choice The short-run supply curve for a perfectly competitive firm is its marginal cost curve
        above the minimum point on the AVC (average variable cost) curve.
        \choice The marginal cost curve does not give the quantity that a perfectly competitive firm
        wants to produce at a given price.
        \choice The short-run supply curve for a perfectly competitive firm is horizontal.
        \choice A perfectly competitive firm does not have a supply curve.
    \end{choices}
\end{question}

\begin{explanation}
    For a perfectly competitive firm (price taker):
    \begin{itemize}
        \item The demand is horizontal at the market price: $D^f=P^e$.
        \item The supply is the portion of AVC curve above the minimum point.
    \end{itemize}
    B is incorrect because the marginal cost curve does give the quantity
    that a perfectly competitive firm wants to produce at a given price: $P=MC$.

    C is incorrect because the \textit{demand} curve is horizontal, not the supply curve.

    D is incorrect because only a \textit{firm with market power} (monopoly, monopolistic competition, oligopoly) \textit{does not have a supply curve}.
\end{explanation}

\begin{question}
    When Marie was a DTU student with no income, she was used to taking the bus for daily
    commuting. After graduation, Marie gets a job. As such, her income is now \num{400000} DKK a
    year. Now she stops taking the bus and instead she drives a car for daily commuting. In
    economic terms, bus transport is a(n) \blank, while car driving is a(n) \blank.
    \begin{choices}
        \choice \textbf{inferior good; normal good.}
        \choice normal good; inferior good.
        \choice normal good; normal good.
        \choice inferior good; inferior good.
    \end{choices}
\end{question}

\begin{question}
    A firm uses labor (L) and capital (K) as inputs to produce. If the price of the two inputs are $w = \SI{120}{DKK}, r = \SI{210}{DKK}$,
    and marginal products are $MPL = 40, MPK = 70$, the firm:
    \begin{choices}
        \choice should use less K and more L to minimize cost.
        \choice is profit maximizing but not cost minimizing.
        \choice \textbf{is cost minimizing.}
        \choice should use more L and less K to minimize cost.
    \end{choices}
\end{question}

\begin{explanation}
    We see that it is already cost minimizing, since
    $$\frac{MP_L}{w} = \frac{MP_K}{r}$$
    See earlier question.
\end{explanation}

\begin{question}
    Which of the following is a profit-maximizing condition for a Cournot oligopolist?
    \begin{choices}
        \choice $\boxed{MR = MC}$.
        \choice $Q_1 = Q_2 = \ldots = Q_n$ \textit{(equilibrium)}
        \choice $P = MR$ \textit{(price-taker)}
        \choice All of the statements associated with this question are correct.
    \end{choices}
\end{question}

\begin{explanation}
    In a Cournot oligopoly, firms react to each other's profit-maximizing output quantities:
    $$Q_1 = r_1(Q_2, \ldots, Q_n) \quad\quad \cdots \quad\quad Q_n = r_n(Q_1, \ldots, Q_{n-1})$$
    The \textit{best response} or \textit{reaction function} captures the relationship between
    each firm's profit-maximizing output.

    Each firm makes an output decision based on the output level under the belief
    that its rival will hold its output constant when the other firm changes its output.
    Thus, each firm maximizes its profit given the other firms' outputs, so $MR = MC$.
    We derive that A is correct and C is not.

    The Cournot \textit{equilibrium} is the intersection of the reaction functions of the firms,
    where neither firm has an incentive to change its output given the other firm's output. In this case,
    $$Q_1 = Q_2 = \ldots = Q_n$$
    We thus derive that B also not correct.
\end{explanation}

\begin{question}
    When firm 1 enjoys a first-mover advantage in a Stackelberg duopoly, it will produce:
    \begin{choices}
        \choice more output and charge a lower price than firm 2.
        \choice \textbf{more output and charge the same price as firm 2.}
        \choice less output and charge the same price as firm 2.
        \choice less output and charge a higher price than firm 2.
    \end{choices}
\end{question}

\begin{explanation}
    A Stackelberg oligopoly follows a \textbf{leader-follower} structure, where the leader
    (firm 1) makes its output decision first, and the follower (firm 2) makes its output
    decision second, taking the leader's output as given.

    Given a linear demand function and cost functions
    $$P = a - b(Q_1 + Q_2), \qquad C_1(Q_1) = c_1 \cdot Q_1, \qquad C_2(Q_2) = c_2 \cdot Q_2$$
    the follower takes the leader's output $Q_1$ and adjusts its output quantity:
    $$Q_2 = r_2(Q_1) = \frac{a-c_2}{2b} - \frac{Q_1}{2}$$
    The leader produces
    $$Q_1 = \frac{a+c_2-2c_1}{2b}$$
    The leader strategically chooses a high level of output to maximize its own profit, forcing the follower to produce less.
    Since the total output $Q = Q_1 + Q_2$ sets \textbf{one market price}, both charge the same price.
\end{explanation}

\begin{question}
    Two identical firms compete as a Cournot duopoly. The demand they face is $P = 100 - 2Q$. The cost
    function for each firm is $C(Q) = 4Q$. The equilibrium output of each firm is:
    \begin{choices}
        \choice 8.
        \choice \textbf{16.}
        \choice 32.
        \choice 36.
    \end{choices}
\end{question}

\begin{explanation}
    The costs are
    $$C_1(Q) = 4Q = C_2(Q) \implies MC_1(Q) = MC_2(Q) = 4$$
    The (inverse) demand is
    $$P = 100 - 2Q = 100 - 2(Q_1 + Q_2)$$
    Each firm sets $MR_1(Q_1) = MC_1(Q_1)$ and $MR_2(Q_2) = MC_2(Q_2)$.
    For firm one we have:
    \begin{align*}
        R_1(Q_1)  & = P(Q_1 + Q_2)           &  & = (100 - 2(Q_1 + Q_2))Q_1 \\
        MR_1(Q_1) & = \frac{dR_1(Q_1)}{dQ_1} &  & = 100 - 4Q_1 - 2Q_2
    \end{align*}
    And we get equilibrium when
    \begin{align*}
        MR_1(Q_1) = MC_1(Q_1) & \iff 100 - 4Q_1 - 2Q_2 = 4 \\
                              & \iff 96 - 2Q_2 = 4Q_1
    \end{align*}
    Since $Q_1 = Q_2$ we get
    \begin{align*}
        Q_1 = Q_2 = \frac{96 - 2Q_2}{4} & \iff 4Q_2 = 96 - 2Q_2 \\
                                        & \iff 6Q_2 = 96        \\
                                        & \iff Q_2 = 16
    \end{align*}
\end{explanation}

\begin{question}
    Two identical firms compete as a Cournot duopoly. The demand they face is $P = 100 - 2Q$. The cost
    function for each firm is $C(Q) = 4Q$. Each firm earns equilibrium profits of:
    \begin{choices}
        \choice 1,024.
        \choice 2,048.
        \choice 4,096.
        \choice \textbf{512.}
    \end{choices}
\end{question}

\begin{explanation}
    We already calculated that the equilibrium output of each firm is
    $$Q_1 = Q_2 = 16$$
    The price is
    $$P = 100-2Q = 100 - 2(\red{Q_1 + Q_2}) = 100 - 2(16 + 16) = 36$$
    The profit for each firm is
    $$\pi_1 = \pi_2 = P \cdot Q_1 - C_1(Q_1) = (P - MC(Q_1))Q_1 = (36 - 4) \cdot 16 = 512$$
\end{explanation}

\begin{question}
    Two firms compete as a Stackelberg duopoly. The demand they face is $P = 100 - 3Q$. The cost
    function for each firm is $C(Q) = 4Q$. The outputs of the two firms are:
    \begin{choices}
        \choice $\boxed{Q_L = 16, \, Q_F = 8}$
        \choice $Q_L = 24, \, Q_F = 12.$
        \choice $Q_L = 12, \, Q_F = 8.$
        \choice $Q_L = 20, \, Q_F = 15.$
    \end{choices}
\end{question}

\begin{explanation}
    We have that $$P = a - bQ = 100 - 3Q = \underbrace{100}_{a} - \underbrace{3}_{b}(Q_1 + Q_2), \quad\quad c_1 = c_2 = 4$$
    Thus, the leader sets its output level at
    $$Q_L = Q_1 = \frac{a + c_2 - 2c_1}{2b} = \frac{100 + 4 - 2 \cdot 4}{2 \cdot 3} = \frac{96}{6} = 16$$
    The follower then sets its output level at
    $$Q_F = Q_2 = r_2(Q_1) = \frac{a - c_2}{2b} - \frac{Q_1}{2} = \frac{100 - 4}{2 \cdot 3} - \frac{16}{2} = \frac{96}{6} - 8 = 16 - 8 = 8$$
\end{explanation}

\begin{question}
    Two firms compete as a Stackelberg duopoly. The demand they face is $P = 100 - 3Q$. The cost
    function for each firm is $C(Q) = 4Q$. The profits of the two firms are:
    \begin{choices}
        \choice $\boxed{\pi_L = 384,\, \pi_F = 192}$
        \choice $\pi_L = 192,\, \pi_F = 91.$
        \choice $\pi_L = 56,\, \pi_F = -28.$
        \choice $\pi_L = 56,\, \pi_F = 28.$
    \end{choices}
\end{question}

\begin{explanation}
    We already calculated that $Q_L = 16$ and $Q_F = 8$. The price is
    $$P = 100-3Q = 100 - 3(Q_1 + Q_2) = 100 - 3(16 + 8) = 28$$
    The profit for each firm is
    \begin{align*}
        \pi_L & = P \cdot Q_L - C_L(Q_L) & = (P - MC(Q_L))Q_L & = (28 - 4) \cdot 16 & = 384 \\
        \pi_F & = P \cdot Q_F - C_F(Q_F) & = (P - MC(Q_F))Q_F & = (28 - 4) \cdot 8  & = 192
    \end{align*}
\end{explanation}

\begin{question}
    Which would you expect to make the highest profits, other things equal?
    \begin{choices}
        \choice Bertrand oligopolist
        \choice Cournot oligopolist
        \choice \textbf{Stackelberg leader}
        \choice Stackelberg follower
    \end{choices}
\end{question}

\begin{explanation}
    \textbf{1. Stackelberg Leader (Highest Profit)}
    The leader moves first and anticipates the follower's reaction.
    $$ Q_L = \frac{a-c}{2b} \quad \text{and} \quad {\pi_L = \frac{(a-c)^2}{8b}} $$

    \textbf{2. Cournot Oligopolist (Middle Profit)}
    Firms move simultaneously. The output is lower than the leader's but higher than the follower's.
    $$ Q_C = \frac{a-c}{3b} \quad \text{and} \quad \pi_C = \frac{(a-c)^2}{9b} $$

    \textbf{3. Stackelberg Follower (Low Profit)}
    The follower reacts to the leader's high output by restricting their own production.
    $$ Q_F = \frac{a-c}{4b} \quad \text{and} \quad \pi_F = \frac{(a-c)^2}{16b} $$

    \textbf{4. Bertrand Oligopolist (Zero Profit)}
    Firms compete on price with homogeneous goods, driving price down to marginal cost.
    $$ P = c \quad \text{and} \quad \boxed{\red{\pi_B = 0}}$$

    \textbf{Conclusion:}
    Comparing the coefficients of the profit terms $\frac{1}{8} > \frac{1}{9} > \frac{1}{16} > 0$, we find:
    $$ \pi_{\text{Stackelberg Leader}} > \pi_{\text{Cournot}} > \pi_{\text{Stackelberg Follower}} > \pi_{\text{Bertrand}} $$
\end{explanation}

With \textit{unequal} marginal costs ($c_1$ for Leader, $c_2$ for Follower), the profit hierarchy depends on the efficiency gap, but the structural advantage of leadership remains.

\textbf{1. Stackelberg Leader (Firm 1)}
The leader optimizes against the follower's reaction function.
$$ Q_1 = \frac{a - 2c_1 + c_2}{2b} \quad \text{and} \quad \pi_1 = \frac{(a - 2c_1 + c_2)^2}{8b} $$

\textbf{2. Stackelberg Follower (Firm 2)}
The follower faces a residual demand curve reduced by the leader's output.
$$ Q_2 = \frac{a - 3c_2 + 2c_1}{4b} \quad \text{and} \quad \pi_2 = \frac{(a - 3c_2 + 2c_1)^2}{16b} $$

\textbf{3. Comparison with Cournot}
If Firm 1 were instead a Cournot competitor, its profit would be:
$$ \pi_{C1} = \frac{(a - 2c_1 + c_2)^2}{9b} $$

\textbf{Conclusion on Profitability:}
Comparing the denominators, we see that $\frac{1}{8b} > \frac{1}{9b}$.
Therefore, \textbf{Firm 1 always prefers being a Stackelberg Leader over being a Cournot competitor} ($\pi_{Stackelberg} > \pi_{Cournot}$), regardless of costs.

However, whether the Leader (Firm 1) earns more than the Follower (Firm 2) depends on their cost difference. If $c_1 \approx c_2$, the Leader wins. If $c_1 \gg c_2$, the efficient Follower may earn higher profits despite the second-mover disadvantage.

\begin{question}
    The Sweezy model of oligopoly reveals that:
    \begin{choices}
        \choice capacity constraints are not important in determining market performance.
        \choice perfectly competitive prices can arise in markets with only a few firms.
        \choice \textbf{changes in marginal cost may not affect prices.}
        \choice All of the statements associated with this question are correct.
    \end{choices}
\end{question}

\begin{explanation}
    A firm only follows another firm if that other firm lowers its price below the prevailing level.
\end{explanation}

\begin{question}
    The market demand in a Bertrand duopoly is $P = 15 - 4Q$, and the marginal costs are $3$. Fixed
    costs are zero for both firms. Which of the following statement(s) is/are true?
    \begin{choices}
        \choice $\boxed{P = 3}$
        \choice $P = 10$
        \choice $P = 15$
        \choice None of the answers is correct.
    \end{choices}
\end{question}

\begin{explanation}
    For a Bertrand oligopoly, price wars come to an end when
    $$P_1 = P_2 = MC$$
    In this case, $P = MC = 3$.
\end{explanation}

\begin{question}
    Profits $\pi$ are higher as isoprofit curves move closer to the:
    \begin{choices}
        \choice \textbf{monopoly output}, $Q^M$
        \choice Cournot output, $Q^C$
        \choice Bertrand output, $Q^B$
        \choice peak of each isoprofit curve.
    \end{choices}
\end{question}

\begin{question}
    In the game shown below, firms 1 and 2 must independently
    decide whether to charge high or low prices.
    \begin{center}
        \vspace{0.2cm}
        \renewcommand{\arraystretch}{1.5}
        \begin{tabular}{l||c|c}
                         & Firm 2: High & Firm 2: Low \\
            \hline \hline
            Firm 1: High & $(10,10)$    & $(5,-5)$    \\
            \hline
            Firm 1: Low  & $(5,-5)$     & $(0,0)$     \\
        \end{tabular}
        \vspace{0.2cm}
    \end{center}
    Which one of the following are Nash equilibrium payoffs in the one-shot game?
    \begin{choices}
        \choice $(0,0)$
        \choice $(5,-5)$
        \choice $(-5,5)$
        \choice $\boxed{(10,10)}$
    \end{choices}
\end{question}

\begin{explanation}
    This is a normal-form game where each cell represents
    $$(\alpha, \beta) = (\text{payoff}_1, \text{payoff}_2)$$
    A Nash equilibrium is a set of strategies in which no player can improve their payoff
    by \textit{unilaterally} changing their strategy, \textit{given} the strategies of the \textit{other} players.

    \begin{itemize}
        \item \textbf{Player 1}: Choose a cell. Check if reward for \red{player 1}\, improves by changing strategy (\red{row}) within the current \red{column} (strategy of Player 2).
        \item \textbf{Player 2}: Choose a cell. Check if reward for \red{player 2}\, improves by changing strategy (\red{column}) within the current \red{row} (strategy of Player 1).
    \end{itemize}

    We see that only $(10,10)$ is a Nash equilibrium. This also happens to be the
    joined optimum, but it doesn't have to be.
\end{explanation}

\begin{question}
    Which of the following are the Nash equilibrium payoffs (\textbf{each period}) if the game described above is repeated 10
    times? I.e. a 10-shot game?
    \begin{choices}
        \choice $(0,0)$
        \choice $(5,-5)$
        \choice $(-5,5)$
        \choice $\boxed{(10,10)}$
    \end{choices}
\end{question}

\begin{explanation}
    The one-shot game from above is the last stage (stage 10) of a 10-shot game.
    The reward was the Nash equilibrium, namely $(10,10)$.
    Now we work backwards from stage 10 to stage 1, creating new normal form
    tables where we add the payoffs from stage $\ell$ to stage $\ell - 1$,
    discounted by $\delta$. Thereby we get an aggregated matrix.

    We add $(10,10)$ to each cell of the original one-shot game to get the matrix for stage $10-1=9$:
    \begin{center}
        \vspace{0.2cm}
        \renewcommand{\arraystretch}{1.5}
        \begin{tabular}{l||c|c}
                         & Firm 2: High & Firm 2: Low \\
            \hline \hline
            Firm 1: High & $(20,20)$    & $(15,5)$    \\
            \hline
            Firm 1: Low  & $(15,5)$     & $(10,10)$   \\
        \end{tabular}
        \vspace{0.2cm}
    \end{center}
    We see that in each stage upperleft, i.e. $(10,10)$, is a Nash equilibrium.
\end{explanation}

\begin{question}
    Consider the following entry game: Here, firm B is an existing firm in the market, and firm A is a
    potential entrant. Firm A must decide whether to enter the market (play ``enter'') or stay out of the
    market (play ``not enter''). If firm A decides to enter the market, firm B must decide whether to
    engage in a price war (play ``hard''), or not (play ``soft''). By playing ``hard'', firm B ensures that firm A
    makes a loss of \$1 million, but firm B only makes \$1 million in profits. On the other hand, if firm B
    plays ``soft'', the new entrant takes half of the market, and each firm earns profits of \$5 million. If
    firm A stays out, it earns zero while firm B earns \$10 million. Which of the following are Nash
    equilibrium strategies?
    \begin{choices}
        \item (enter, hard) and (not enter, hard)
        \item (enter, soft) and (not enter, soft)
        \item \textbf{(not enter, \red{hard}) and (enter, soft)}
        \item (enter, hard) and (not enter, soft)
    \end{choices}
\end{question}

\begin{explanation}
    This is an extensive-form game, where players make decision sequentially,
    often represented by a game tree.

    However, to solve for the Nash equilibria, it is helpful to convert it to a normal-form game.
    \begin{center}
        \vspace{0.2cm}
        \renewcommand{\arraystretch}{1.5}
        \begin{tabular}{l||c|c}
                              & Firm 2: Hard & Firm 2: Soft \\
            \hline \hline
            Firm 1: Enter     & $(-1,1)$     & $(5,5)$      \\
            \hline
            Firm 1: Not Enter & $(0,10)$     & $(0,10)$     \\
        \end{tabular}
        \vspace{0.2cm}
    \end{center}
    We now see that the Nash equilibria are $(5,5)$ and $(0,10)$
    (for \red{\textit{hard}}, because for \textit{soft}, Firm 1 can improve its payoff by entering).
\end{explanation}

\begin{question}
    The figure below presents information for a one-shot game.
    \begin{center}
        \vspace{0.2cm}
        \renewcommand{\arraystretch}{1.5}
        \begin{tabular}{l||c|c}
                               & Firm 2: Low price & Firm 2: High price \\
            \hline \hline
            Firm 1: Low price  & $(2,2)$           & $(10,-8)$          \\
            \hline
            Firm 1: High price & $(-8,10)$         & $(6,6)$            \\
        \end{tabular}
        \vspace{0.2cm}
    \end{center}
    What are secure strategies for firm A and firm B respectively?
    \begin{choices}
        \choice (low price, high price)
        \choice (high price, low price)
        \choice (high price, high price)
        \choice (low price, low price)
    \end{choices}
\end{question}

\begin{explanation}
    A \textbf{secure strategy} is a strategy that guarantees the highest (minimum) payoff to a player,
    given the \textit{worst possible scenario}

    Firm 1's perspective:
    \begin{itemize}
        \item Row 1 = low: minimum payoff for Firm 1 is 2
        \item Row 2 = high: minimum payoff for Firm 1 is -8
    \end{itemize}
    $\implies$ Firm 1's secure strategy is \textit{low}, since this yields the \textbf{highest minimum payoff} over \textbf{rows}.

    Firm 2's perspective:
    \begin{itemize}
        \item Column 1 = low: minimum payoff for Firm 2 is 2
        \item Column 2 = high: minimum payoff for Firm 2 is -8
    \end{itemize}
    $\implies$ Firm 2's secure strategy is \textit{low}, since this yields the \textbf{highest minimum payoff} over \textbf{columns}.
\end{explanation}

\begin{question}
    What are the dominant strategies for firm A and firm B respectively in the game above?
    \begin{choices}
        \choice (low price, high price)
        \choice (high price, low price)
        \choice (high price, high price)
        \choice \textbf{(low price, low price)}
    \end{choices}
\end{question}

\begin{explanation}
    A \textbf{dominant strategy} results in the highest payoff to a player regardless
    of the opponent's actions.

    Firm 1's perspective:
    \begin{itemize}
        \item Firm 2 plays low: Firm 1 prefers low (2) over high (-8)
        \item Firm 2 plays high: Firm 1 prefers low (10) over high (6)
    \end{itemize}
    $\implies$ Firm 1's dominant strategy is (low price)\\
    $\implies$ The strategy must be \textbf{the same for all actions of Firm 2!}

    Firm 2's perspective:
    \begin{itemize}
        \item Firm 1 plays low: Firm 2 prefers low (2) over high (-8)
        \item Firm 1 plays high: Firm 2 prefers low (10) over high (6)
    \end{itemize}
    $\implies$ Firm 2's dominant strategy is (low price)

    We conclude that (low price, low price) are the dominant strategies.
\end{explanation}

\begin{question}
    Refer to the following game.
    \begin{center}
        \vspace{0.2cm}
        \renewcommand{\arraystretch}{1.5}
        \begin{tabular}{l||c|c}
                               & Firm 2: Low price & Firm 2: High price \\
            \hline \hline
            Firm 1: Low price  & $(10,9)$          & $(15,8)$           \\
            \hline
            Firm 1: High price & $(-10,7)$         & $(11,11)$          \\
        \end{tabular}
        \vspace{0.2cm}
    \end{center}
    Which of the following is true?
    \begin{choices}
        \choice A dominant strategy for Firm 1 is ``high price''
        \choice There does not exist a dominant strategy for firm A.
        \choice A dominant strategy for Firm 2 is ``low price''
        \choice \textbf{None of the answers is correct.}
    \end{choices}
\end{question}

\begin{explanation}
    We look at the dominant strategies for each firm.

    Firm 1's perspective:
    \begin{itemize}
        \item Firm 2 plays low: Firm 1 prefers low (10) over high (-10)
        \item Firm 2 plays high: Firm 1 prefers low (15) over high (11)
    \end{itemize}
    $\implies$ Firm 1's dominant strategy is (low price)

    Firm 2's perspective:
    \begin{itemize}
        \item Firm 1 plays low: Firm 2 prefers low (9) over high (7)
        \item Firm 1 plays high: Firm 2 prefers high (11) over low (7)
    \end{itemize}
    $\implies$ Firm 2 \textit{does not have} a dominant strategy.
\end{explanation}

\begin{question}
    Refer to the following game.
    \begin{center}
        \vspace{0.2cm}
        \renewcommand{\arraystretch}{1.5}
        \begin{tabular}{l||c|c|c}
                         & Player 2: t1 & Player 2: t2 & Player 2: t3 \\
            \hline \hline
            Player 1: S1 & $(4,10)$     & $(3,0)$      & $(1,3)$      \\
            \hline
            Player 1: S2 & $(0,0)$      & $(2,10)$     & $(10,3)$     \\
        \end{tabular}
        \vspace{0.2cm}
    \end{center}
    Which of the following strategies constitutes a Nash equilibrium?
    \begin{choices}
        \choice \textbf{S1, t1}
        \choice S2, t2
        \choice S2, t3
        \choice S1, t2
    \end{choices}
\end{question}

\begin{question}
    Refer to the following game.
    \begin{center}
        \vspace{0.2cm}
        \renewcommand{\arraystretch}{1.5}
        \begin{tabular}{l||c|c|c}
                         & Player 2: t1 & Player 2: t2 & Player 2: t3 \\
            \hline \hline
            Player 1: S1 & $(10,0)$     & $(5,1)$      & $(4,-200)$   \\
            \hline
            Player 1: S2 & $(10,100)$   & $(5,0)$      & $(0,-100)$   \\
        \end{tabular}
        \vspace{0.2cm}
    \end{center}
    Which of the following pairs of strategies constitute a Nash equilibrium of the game?
    \begin{choices}
        \choice S1, t1
        \choice S1, t2
        \choice S2, t1
        \choice \textbf{S1, t2 and S2, t1}
    \end{choices}
\end{question}

\begin{question}
    It is easier to sustain tacit collusion in an infinitely repeated game if:
    \begin{choices}
        \choice \textbf{the present value of cheating is lower than collusion.}
        \choice there are many players.
        \choice the interest rate is higher.
        \choice the present value of cheating is lower than collusion and the interest rate is higher.
    \end{choices}
\end{question}

\begin{question}
    A coordination problem usually occurs in situations where there is:
    \begin{choices}
        \choice no Nash equilibrium in a game.
        \choice a unique, but undesirable Nash equilibrium.
        \choice a unique, secure strategy for both players.
        \choice \textbf{more than one Nash equilibrium.}
    \end{choices}
\end{question}

\begin{question}
    Suppose $P = 20 - 2Q$ is the market demand function for a local monopoly. The marginal cost is $2Q$.
    The local monopoly tries to maximize its profits by equating $MC = MR$ and charging a uniform price.
    What will be the equilibrium price and output?
    \begin{choices}
        \choice $\$6.33, 3.33$ units
            \choice $\$6.33, 5$ units
            \choice $\boxed{\$13.33, 3.33 \text{ units}}$
            \choice $\$10, 5$ units
    \end{choices}
\end{question}

\begin{explanation}
    The monopoly will produce $Q^M$ where $MC = MR$:
    $$MC = MR \iff 2Q^M = 20 - 2 \cdot 2 Q^M \iff Q^M = \frac{20}{6} \approx 3.33$$
    In this case, the price is:
    $$P^M = P(Q^M) = 20 - 2Q^M = 20 - 2 \cdot \frac{20}{6} = \frac{60}{3} - \frac{20}{3} = \frac{40}{3} \approx \$13.33$$
\end{explanation}

\begin{question}
    Consider the same scenario as the previous question.
    Which of the following will allow the monopolistic firm to enhance the profits?
    \begin{choices}
        \choice \textbf{Engage in two-part pricing.}
        \choice Engage in commodity bundling.
        \choice Engage in randomized pricing.
        \choice Engage in two-part pricing and engage in commodity bundling.
    \end{choices}
\end{question}

\begin{explanation}
    Under the current \textit{uniform} pricing strategy,
    the firm is leaving money on the table in two ways:
    \begin{enumerate}
        \item \textbf{Consumer Surplus}: Some customers are willing to pay more than \$13.33 but don't have to.
        \item \textbf{Deadweight Loss}: There are customers willing to pay between \$10 (the marginal cost) and \$13.33 (the price).
              Since the cost to produce is lower than what they are willing to pay, these are \textit{missed sales}.
    \end{enumerate}
    Recall that the Deadweight Loss is:
    \begin{itemize}
        \item Total surplus lost by society (both consumers and producers)
        \item Because the monopolist produces less than the socially optimal output
        \item Equivalently, because the monopolist charges a price higher than marginal cost
        \item The triangle between the demand curve, the marginal cost curve, and vertical line through $Q^M$
    \end{itemize}

    A two-part pricing strategy consists of two separate charges:
    \begin{enumerate}
        \item \textbf{Per-Unit Price}: $P$
              \begin{itemize}
                  \item Is set equal to Marginal Cost $MC$: $\boxed{P=MC}$
                  \item Eliminates Deadweight Loss
              \end{itemize}
        \item \textbf{Fixed Fee}: $F$
              \begin{itemize}
                  \item Is set equal to the total Consumer Surplus $CS$: $\boxed{F=CS}$
                  \item Eliminates Consumer Surplus
              \end{itemize}
    \end{enumerate}
\end{explanation}

\begin{question}
    Suppose $P = 20 - 2Q$ is the market demand function for a local monopoly. The marginal cost is 2Q.
    If fixed costs are zero and the firm engages in two-part pricing, the most profits the firm will earn is:
    \begin{choices}
        \choice \$5.
        \choice \$10.
        \choice \$25.
        \choice \$50.
    \end{choices}
\end{question}

\begin{explanation}
    In two part pricing, the firm will set
    $$P = MC \rightsquigarrow \text{find } Q^*, P^*, \qquad \qquad F = CS \red{\neq TS!}$$
    The intersection of $MP$ and the demand curve is the socially optimal output:
    $$P(Q^*) = MC(Q^*) \iff 2Q^* = 20 - 2Q^* \iff Q^* = \frac{20}{4} = 5$$
    In this case, we get that
    \begin{align*}
        P^* & = MC(Q^*) = 2Q^* = 2 \cdot 5                                                           &  & = 10 \\
        F   & = \red{CS} = \frac{1}{2} \cdot (20\red{-P^*}) \cdot Q^* = \frac{1}{2} \cdot 10 \cdot 5 &  & = 25
    \end{align*}
    From $MC(Q) = 2Q$ and $FC = 0$, we get that $C(Q) = Q^2$.\\
    The total profit is now $$\pi = (P^* \cdot Q^*) - C(Q^*) + F = (10 \cdot 5) - (5^2) + 25 = 50$$.
\end{explanation}

\begin{question}
    Which of the following statements is true?
    \begin{choices}
        \choice The more elastic the demand, the higher the profit-maximizing markup.
        \choice \textbf{The more elastic the demand, the lower the profit-maximizing markup.}
        \choice The higher the marginal cost, the lower the profit-maximizing price.
        \choice The higher the average cost, the lower the profit-maximizing price.
    \end{choices}
\end{question}

\begin{explanation}
    A monopoly and a monopolistically competitive firm use the profit-maximizing price:
    $$\boxed{P = \underbrace{\left(\frac{E_F}{1+E_F}\right)}_{\textbf{Markup}}MC}$$
    Markup is the amount by which the selling price of a product exceeds its cost to produce.
    It is the ``premium'' a firm charges to cover overhead (rent, salaries) and generate a profit.

    In the above formula, $$E_F = E_{Q_X^d, P_X^d} = \frac{\Delta \% Q_X^d}{\Delta \% P_X^d}$$
    is the price elasticity of demand.\\
    Let's try two examples:
    \begin{itemize}
        \item \textbf{Case 1: Low Elasticity ($E_F = -2$)}
              $$ P = \left(\frac{-2}{-2+1}\right) MC = \left(\frac{-2}{-1}\right) MC = 2 \cdot MC $$
              The markup factor is \textbf{2.0} (100\% markup).

        \item \textbf{Case 2: High Elasticity ($E_F = -5$)}
              $$ P = \left(\frac{-5}{-5+1}\right) MC = \left(\frac{-5}{-4}\right) MC = 1.25 \cdot MC $$
              The markup factor is \textbf{1.25} (25\% markup).
    \end{itemize}

    \textbf{Conclusion:} As demand becomes \textbf{more elastic} (consumers become more price-sensitive), the firm loses pricing power, causing the markup to \textbf{decrease}.
\end{explanation}

\begin{question}
    During spring break, students have an elasticity of demand for a trip to Florida of -3. How much
    should an airline charge students for a ticket if the price it charges the general public is \$360?
    Assume the general public has an elasticity of -2.
    \begin{choices}
        \choice \$240
        \choice \$250
        \choice \$260
        \choice \textbf{\$270}
    \end{choices}
\end{question}

\begin{explanation}
    For the general public, we have:
    $$P = \left(\frac{-2}{-2+1}\right) MC \iff 360 = \left(\frac{-2}{-1}\right) MC \iff MC = \frac{360}{2} = 180$$
    Now we calculate the price for students:
    $$P = \left(\frac{-3}{-3+1}\right) MC = \left(\frac{-3}{-2}\right) 180 = \frac{540}{2} = 270$$
\end{explanation}

\begin{question}
    What price should a firm charge for a package of two shirts given a marginal cost of \$2 and an
    inverse demand function $P = 6 - 2Q$ by the representative consumer?
    \begin{choices}
        \choice \$2
        \choice \$6
        \choice \textbf{\$8}
        \choice \$10
    \end{choices}
\end{question}

\begin{explanation}
    In \textit{block pricing}, we have an ``all-or-nothing'' deal.

    \begin{enumerate}
        \item Determine the social optimum quantity $Q^*$:
              $$P(Q^*) = MC(Q^*) \iff 6 - 2Q^* = 2 \iff Q^* = \frac{6-2}{2} = 2$$
        \item Determine the social optimum price $P^*$:
              $$P^* = P(Q^*) = 6 - 2 \cdot 2 = 2$$
        \item Because in this case, MC is constant, we have immediately that $P^* = 2$,
              so the above was a waste of time.
        \item Set the package price to be the total value the consumer receives for purchasing $Q^*$ units:
              $$\boxed{\text{Package Price} = \int_{0}^{Q^*} P(Q) \, dQ }= \int_{0}^{2} (6 - 2Q) \, dQ = \left[6Q - Q^2\right]_{0}^{2} = 12 - 4 = 8$$
        \item The profit in this case
              $$\boxed{\pi = \text{Package Price} - MC(Q^*) \cdot Q^*} = 8 - 2 \cdot 2 = 4$$
    \end{enumerate}
\end{explanation}

\begin{question}
    The special cost structure that is necessary for a firm to adopt a peak-load pricing policy is:
    \begin{choices}
        \choice economies of scale.
        \choice economies of scope.
        \choice constant marginal cost.
        \choice \textbf{limited capacity}.
    \end{choices}
\end{question}

\begin{explanation}
    Peak-load pricing involves charging higher prices during periods of peak demand
    and lower prices during periods of off-peak demand.

    It requires \textbf{limited capacity} because supply cannot be easily increased
    in the short run to meet demand spikes.

    \begin{itemize}
        \item During \textbf{off-peak} times, demand is below capacity.
              The price is set equal to the marginal operating cost ($P = MC$).
        \item During \textbf{peak} times, demand hits the physical capacity limit ($Q_{max}$).
              The price must be raised to ration the scarce supply and cover the capital costs of
              building that capacity.
    \end{itemize}

    If capacity were unlimited, the firm would simply produce more output at the standard marginal cost, eliminating the need for a price hike.
\end{explanation}

\begin{question}
    Which group of policies aims at discouraging rivals from starting a price war?
    \begin{choices}
        \choice \textbf{price matching and randomized pricing.}
        \choice price matching, brand loyalty, and commodity bundling.
        \choice randomized pricing, price discrimination, and cross-subsidization.
        \choice peak-peak pricing, two-part pricing, and price matching.
    \end{choices}
\end{question}

\begin{explanation}
    The correct choice is \textbf{price matching and randomized pricing}. These strategies specifically alter the strategic interaction between firms to make price-cutting less attractive.

    \begin{itemize}
        \item \textbf{Price Matching:} A firm publicly announces it will match any lower price offered by a rival. This removes the rival's incentive to undercut. If the rival lowers their price, they won't steal any customers (because the first firm matches instantly), but they will earn lower margins. Thus, the rival is discouraged from starting a price war.
        \item \textbf{Randomized Pricing:} By constantly varying prices, a firm makes it difficult for rivals to predict its price. If a rival cannot be sure of the competitor's price, they cannot set a price just slightly lower to steal the market. This obscurity reduces the incentive to engage in predatory price cutting.
    \end{itemize}

    \textbf{Why the others are incorrect:}
    \begin{itemize}
        \item \textit{Commodity bundling} and \textit{two-part pricing} are primarily methods to extract consumer surplus, not to manage rival behavior.
        \item \textit{Price discrimination} is about charging different prices to different consumers based on willingness to pay, not deterring competitors.
    \end{itemize}
\end{explanation}

\begin{question}
    Suppose two types of consumers buy suits. Consumers of type A will pay \$100 for a coat and \$50
    for pants. Consumers of type B will pay \$75 for a coat and \$75 for pants. The firm selling suits faces
    no competition and has a marginal cost of zero. If the firm charges \$100 for a suit (which includes
    both pants and a coat), the firm will sell a suit to:
    \begin{choices}
        \choice type A consumers.
        \choice type B consumers.
        \choice \textbf{type A consumers and type B consumers.}
        \choice None of the answers are correct.
    \end{choices}
\end{question}

\begin{explanation}
    To determine who buys the suit, we must calculate the total value each consumer places on the bundle (Coat + Pants).

    \begin{center}
        \renewcommand{\arraystretch}{1.5}
        \begin{tabular}{l||c|c||c}
            Consumer Type & WTP coat & WTP pants & \textbf{Total WTP (Suit)} \\ \hline\hline
            Type A        & \$100    & \$50      & \$150                     \\ \hline
            Type B        & \$75     & \$75      & \$150                     \\
        \end{tabular}
    \end{center}

    The firm charges \textbf{\$100} for the suit.
    \begin{itemize}
        \item \textbf{Type A:} Values the suit at \$150. Since $\$150 \ge \$100$, they will buy it (and enjoy a consumer surplus of \$50).
        \item \textbf{Type B:} Values the suit at \$150. Since $\$150 \ge \$100$, they will also buy it (and enjoy a consumer surplus of \$50).
    \end{itemize}

    Therefore, \textbf{both type A and type B consumers} will purchase the suit.
\end{explanation}

\begin{question}
    In the same scenario as above, the optimal commodity bundling strategy is:
    \begin{choices}
        \choice \textbf{Charge \$150 for a suit.}
        \choice Charge \$75 for a suit.
        \choice Charge \$100 for a suit.
        \choice Charge \$125 for a suit.
    \end{choices}
\end{question}

\begin{question}
    \blank occurs when people smoke more after buying life insurance.
    \begin{choices}
        \choice Adverse selection
        \choice \textbf{Moral hazard}
        \choice Asymmetric information
        \choice Cournot and Bertrand competition
    \end{choices}
\end{question}

\begin{explanation}
    \textbf{Adverse selection} refers to (asymmetric information) situations where
    individuals have hidden characteristics and in
    which a selection process results in a pool of
    individuals with undesirable characteristics.\\
    $\quad \rightarrow$ \textbf{before} the deal\\
    $\quad \rightarrow$ hidden \textbf{information/characteristics}

    \textit{Example:} The car market. Sellers of used cars know if their car is of bad quality or good quality,
    but buyers do not. Since buyers offer an average price, owners of good cars refuse to sell,
    leaving only bad cars in the market. Another example is unhealthy people being more likely to buy health insurance than healthy people.

    \textbf{Moral hazard} refers to an (asymmetric information) situation where one
    party to a contract takes a hidden action that
    benefits him or her at the expense of another
    party.\\
    $\quad \rightarrow$ \textbf{after} the deal\\
    $\quad \rightarrow$ hidden \textbf{action}

    \textit{Example:} A person who buys full theft insurance for their bicycle
    might stop locking it up because they no longer bear the financial cost of it being stolen.
    Or, as in the question, someone smoking more because they have life insurance.

    It can be mitigated by:
    \begin{itemize}
        \item \textbf{Incentive contracts}
        \item \textbf{Signaling:} (a warranty signals a good car).
        \item \textbf{Screening:} (a medical exam before insurance is a \textit{self-selection device}).
    \end{itemize}
\end{explanation}

\begin{question}
    To maximize profit in the face of uncertainty, firms should produce the output where:
    \begin{choices}
        \choice expected price equals expected marginal cost.
        \choice \textbf{expected marginal revenue equals marginal cost.}
        \choice expected marginal revenue equals expected marginal cost.
        \choice expected price equals marginal cost.
    \end{choices}
\end{question}

\begin{explanation}
    If demand (hence, revenue) is uncertain and the
    manager is risk neutral, then the manager will
    want to maximize expected profits by producing
    the output where the expected marginal
    revenue equals marginal cost:$$\E[MR] = MC$$
\end{explanation}

\begin{question}
    Joe's search costs are \$5 per search. He wants to buy a video player for his wife for Christmas, and
    the lowest price he's found so far is \$300. Joe thinks 80 percent of the stores charge \$300 for
    video players and 20 percent charge \$200. Joe's optimal decision is to:
    \begin{choices}
        \choice \textbf{continue to search for a lower price since the expected benefit of an additional search is
            \$20, which exceeds his per-unit search costs.}
        \choice stop searching and purchase a video player for \$200.
        \choice continue to search for a lower price since the expected benefit of an additional search is
        \$80, which exceeds his per-unit search costs.
        \choice None of the statements is correct.
    \end{choices}
\end{question}

\begin{explanation}
    In order to keep searching, the expected benefit $EB$ of an additional search
    must exceed the search cost $c$:
    $$EB > c$$
    The \textbf{reservation price} $R$ is the price at which a consumer is
    \textit{indifferent} about searching or not searching.
    $$EB[R] = c$$
    The reservation price $R$ is the boundary between the acceptance price region and the rejection price region.

    For this question, the expected benefit of an additional search is
    $$EB = 0.8 \cdot (300 - 300) + 0.2 \cdot (300 - 200) = 0 + \frac{100}{5} = 20$$
    and the search cost is $c = 5$. Since $EB > c$, Joe should continue searching.
\end{explanation}

\begin{question}
    You are a hotel manager and you are considering four projects that yield different payoffs,
    depending upon whether there is an economic boom or a recession. The potential payoffs and
    corresponding payoffs are summarized in the following table.

    \begin{center}
        \vspace{0.2cm}
        \renewcommand{\arraystretch}{1.5}
        \begin{tabular}{l||c|c}
            Project & Boom (50\%) & Recession (50\%) \\ \hline\hline
            A       & 20          & -10              \\ \hline
            B       & -10         & 20               \\ \hline
            C       & 30          & -30              \\ \hline
            D       & 50          & 50               \\
        \end{tabular}
        \vspace{0.2cm}
    \end{center}
    Which project has the greatest expected value?
    \begin{choices}
        \choice
        \choice
        \choice
        \choice $\leftarrow$
    \end{choices}
\end{question}

\begin{explanation}
    The expected value of a project is the sum of the payoffs weighted by their probabilities:
    \begin{align*}
        EV_A & = 0.5 \cdot 20    &  & + 0.5 \cdot (-10) &  & 2 = 5 \\
        EV_B & = 0.5 \cdot (-10) &  & + 0.5 \cdot 20    &  & = 5   \\
        EV_C & = 0.5 \cdot 30    &  & + 0.5 \cdot (-30) &  & = 0   \\
        EV_D & = 0.5 \cdot 50    &  & + 0.5 \cdot 50    &  & = 50
    \end{align*}
    Therefore, project D has the greatest expected value.
\end{explanation}

\begin{question}
    Which of the following phenomena shows that risk aversion is the characteristic of many people?
    \begin{choices}
        \choice Gambling
        \choice Looting
        \choice Investing in one stock rather than a portfolio
        \choice \textbf{Auto insurance}
    \end{choices}
\end{question}

\begin{question}
    An apple farmer must decide how many apples to harvest for the world apple market. He knows
    that there is a one-third probability that the world price will be \$1, a one-third probability that it will
    be \$1.50, and a one-third probability that it will be \$2. His cost function is $C(Q) = 0.01Q^2$. The
    expected profit-maximizing quantity is:
    \begin{choices}
        \choice 0
        \choice 90
        \choice \textbf{75}
        \choice 150
    \end{choices}
\end{question}

\begin{explanation}
    \begin{align*}
        \E[P]  & = \frac{1}{3} \cdot 1 + \frac{1}{3} \cdot \frac{3}{2} + \frac{1}{3} \cdot 2 = \frac{2 + 3 + 4}{6}                                &  & = \frac{9}{6} \\
        \E[MR] & = \E\left[\frac{d}{dQ}(P \cdot Q)\right] = \frac{d}{dQ}\left(\E[P] \cdot Q\right) = \frac{d}{dQ}\left(\frac{9}{6} \cdot Q\right) &  & = \frac{9}{6} \\
        MC(Q)  & = \frac{d}{dQ}(C(Q)) = \frac{d}{dQ}(0.01Q^2)                                                                                     &  & = 0.02Q
    \end{align*}
    We now equate expected marginal revenue to marginal cost:
    \begin{align*}
        \E[MR]       = MC(Q) & \iff \frac{9}{6} = \frac{1}{50}Q                   \\
                             & \iff Q = \frac{9 \cdot 50}{6} = \frac{450}{6} = 75
    \end{align*}
\end{explanation}

\begin{question}
    The optimal bid in a first-price, sealed-bid auction with independent private values is to bid:
    \begin{choices}
        \choice the true value of the item.
        \choice more than the true value of the item.
        \choice \textbf{less than the true value of the item.}
        \choice the true value of the item and more than the true value of the item, depending upon
        whether value estimates are affiliated.
    \end{choices}
\end{question}

\begin{explanation}
    In a Dutch or first-price sealed-bid auction (strategically equivalent) with $n$
    bidders whose private valuations are uniformly distributed between $L$ and $H$,
    the optimal bid $b$ for a player with own valuation $v$ is given by:
    $$b = v - \frac{v - L}{n}$$
\end{explanation}

\begin{question}
    Which of the following auction examples has a common value information structure?
    \begin{choices}
        \choice \textbf{Three firms bid for an oil lease.}
        \choice An auction of a famous painting.
        \choice A college in need of money decides to name a building on campus after the person willing
        to pay the most for the privilege.
        \choice An auction of a famous painting and a college in need of money decides to name a
        building on campus after the person willing to pay the most for the privilege.
    \end{choices}
\end{question}

\begin{explanation}
    The four action types differ in terms of the \textbf{information} in the following ways:
    \begin{enumerate}
        \item The information bidders have about the bids of \textit{other} bidders
              \begin{itemize}
                  \item Perfect information
                  \item Imperfect information
              \end{itemize}
        \item Information structures about the value of their \textit{own} bids
              \begin{itemize}
                  \item \textbf{Independent Private Values:} A bidder's valuation depends on their own individual tastes and preferences. Knowing another bidder's valuation would not change your own.
                        \\ \textit{Example:} The auction of a \textbf{famous painting} or naming rights. One collector may value the painting at \$1M because they love the artist, while another values it at \$0 because they dislike the style.

                  \item \textbf{Affiliated/Common Values:} The true value of the item is the \textit{same} for all bidders (it is objective), but bidders have different \textit{estimates} of that value.
                        \\ \textit{Example:} \textbf{Three firms bidding for an oil lease.} The amount of oil under the ground is a fixed physical fact. If there are 1 million barrels, it is worth roughly the same to Firm A as it is to Firm B. The uncertainty comes from their different geological data (estimates).
              \end{itemize}
    \end{enumerate}

    In the oil lease example, the value is derived from the oil itself (common to all). In contrast, the painting and the building naming rights are driven by subjective prestige or aesthetic preference (private values).
\end{explanation}

\begin{question}
    To avoid the winner's curse, a bidder should:
    \begin{choices}
        \choice not participate in Dutch auctions.
        \choice only participate in second-price and English auctions.
        \choice revise upward his private estimate of the value of the item.
        \choice \textbf{revise downward his private estimate of the value of the item.}
    \end{choices}
\end{question}

\begin{explanation}
    \textbf{Winner's Curse:}
    Common value auctions are subject to the \textit{Winner's Curse}.
    Since the winner is the one who made the highest bid, they are often the one who most
    \textit{overestimated} the true value of the commodity. Therefore, the winner
    frequently ends up losing money (paying more than the item is actually worth).

    \textbf{Dutch Auction:}
    In a Dutch auction, the price starts high and decreases until a bidder accepts the price.

    \textbf{English Auction:}
    In an English auction, the price starts low and increases until a bidder accepts the price.

    To avoid the winner's curve in a common-value auction, a bidder should \textbf{revise downward}
    their private estimate of the value to account for this fact.

    The winner's curse is most pronounced in sealed-bid auctions since bidders don't
    learn about other player's valuation. English auction, in contrast, provides bidders with
    information. Therefore, bidders may have to \textbf{revise up} their
    initial bids.
\end{explanation}

\begin{question}
    John is a seller in an independent private-values auction environment where bidders are risk
    neutral. Which auction yields John the greatest expected revenue?
    \begin{choices}
        \choice English
        \choice First price
        \choice Second price
        \choice \textbf{All of the choices are revenue equivalent.}
    \end{choices}
\end{question}

\begin{explanation}
    In independent private value auctions, the expected revenue is the \textbf{same} for all auctions.

    In affiliated/common value auctions, the revenues are
    $$\text{English} > \text{Second-Price} > \text{First-Price} = \text{Dutch}$$
\end{explanation}

\begin{question}
    Assume that total output consists of 4 apples and 6 oranges and that apples cost \$1 each and oranges cost
    \$0.50 each. In this case, the value of GDP is:
    \begin{choices}
        \choice 10 pieces of fruit.
        \choice \textbf{\$7.}
        \choice \$8.
        \choice \$10.
    \end{choices}
\end{question}

\begin{explanation}
    $$NGDP = P \cdot Y = 4 \times 1 + 6 \times 0.5 = \$7$$
\end{explanation}

\begin{question}
    If nominal GDP increased by 5 percent and the GDP deflator increased by 3 percent, then real GDP \blank
    by \blank percent.
    \begin{choices}
        \choice \textbf{increased; 2}
        \choice decreased; 2
        \choice increased; 8
        \choice decreased; 8
    \end{choices}
\end{question}

\begin{explanation}
    $$\text{GDP Deflator} = 100 \cdot \frac{\text{Nominal GDP}}{\text{Real GDP}}\%$$
    So this means that
    $$\Delta \% \text{GDP Deflator} = \Delta \% \text{Nominal GDP} - \Delta \% \text{Real GDP}$$
    And thus
    $$\Delta \% \text{Real GDP} = 5\% - 3\% = 2\%$$
\end{explanation}

\begin{question}
    GNP equals GDP \blank income earned domestically by foreigners \blank income
    that nationals earn abroad.
    \begin{choices}
        \choice plus; plus
        \choice minus; minus
        \choice \textbf{minus; plus}
        \choice plus; minus
    \end{choices}
\end{question}

\begin{explanation}
    The GNP is the total income earned by the \textbf{nation}'s factors of production, \textbf{regardless of where located}.\\
    $\rightarrow$ does not include income earned domestically by foreigners.\\
    $\rightarrow$ includes income earned abroad by nationals.

    The GDP is the total income earned by \textbf{domestically located} factors of production, \textbf{regardless of nationality}.\\
    $\rightarrow$ includes income earned domestically by foreigners.\\
    $\rightarrow$ does not include income earned abroad by nationals.

    $$GNP - GDP = (\text{factor payments \textbf{from} abroad}) - (\text{factor payments \textbf{to} abroad})$$
    $$GNP = GDP + (\text{factor payments \textbf{from} abroad}) - (\text{factor payments \textbf{to} abroad})$$
\end{explanation}

\begin{question}
    Assume that the consumption function is given by $C = 150 + 0.85(Y - T)$, the tax function is given by
    $T = t_0 + t_1Y$, and $Y = 5,000$. If $t_1$ decreases from 0.3 to 0.2, then consumption increases by:
    \begin{choices}
        \choice 85.
        \choice \textbf{425.}
        \choice 500.
        \choice 525.
    \end{choices}
\end{question}

\begin{explanation}
    \begin{align*}
        \Delta T & = (t_0 + t_1'Y) - (t_0 + t_1Y) = (t_1' - t_1)Y = (0.2 - 0.3) \cdot 5000 &  & = -500 \\
        \Delta C & = 0.85 \cdot (-\Delta T) = 0.85 \cdot 500                               &  & = 425
    \end{align*}
\end{explanation}

\begin{question}
    Assume that the investment function is given by $I = 1000 - 30r$, where $r$ is the real rate of interest (in
    \textbf{percent}). Assume further that the nominal rate of interest is 10 percent and the inflation rate is 2 percent.
    According to the investment function, investment will be:
    \begin{choices}
        \choice 240.
        \choice 700.
        \choice \textbf{760.}
        \choice 970.
    \end{choices}
\end{question}

\begin{explanation}
    The real interest rate $r$ is the nominal interest rate adjusted for inflation:
    $$\text{real interest rate} = r = i - \pi$$
    Where $i$ is the \textit{nominal} interest rate and $\pi$ is the inflation rate.\\
    Substituting the values given in the problem, we get:
    $$r = 10\% - 2\% = 8\%$$
    And thus $I = I(r) = 1000 - 30 \cdot 8 = 760$.
\end{explanation}

\begin{question}
    In the classical model with fixed income, if the interest rate is too high, then investment is too \blank, and
    the demand for output \blank the supply.
    \begin{choices}
        \choice high; exceeds
        \choice high; falls short of
        \choice low; exceeds
        \choice \textbf{low; falls short of}
    \end{choices}
\end{question}

\begin{explanation}
    \textbf{1. Investment is too low}

    Investment depends negatively on the interest rate ($I'(r) < 0$).
    $$ r \uparrow \implies I \downarrow $$
    Therefore, if the interest rate is "too high" (above the equilibrium rate), investment will be lower than required for equilibrium.

    \textbf{2. Demand falls short of Supply}

    We have that the \textbf{aggregate demand} is given by
    $$C(\bar{Y}-\bar{T}) + I(r) + \bar{G}$$
    and that the \textbf{aggregate supply} is given by
    $$\bar{Y} = F(\bar{K},\bar{L})$$
    Thus, if $I(r)$ decreases, then the demand falls short of the supply.
\end{explanation}

\begin{question}
    Assume that equilibrium GDP ($Y$) is 5000. Consumption ($C$) is given by the equation $C = 500 + 0.6Y$. In
    addition, assume $G=0$. In this case, equilibrium investment is:
    \begin{choices}
        \choice \textbf{1500.}
        \choice 2000.
        \choice 2500.
        \choice 3000.
    \end{choices}
\end{question}

\begin{explanation}
    $$\underbrace{Y}_{\text{demand}} = \underbrace{C + I + G}_{\text{supply}}$$
    Substituting the values given in the problem, we get:
    \begin{align*}
        5000 = 500 + 0.6 \cdot 5000 + I(r) & \iff 5000 = 500 + 3000 + I(r) \\
                                           & \iff I(r) = 1500
    \end{align*}
\end{explanation}

\begin{question}
    All of the following are considered major functions of money except as a:
    \begin{choices}
        \choice medium of exchange.
        \choice \textbf{way to display wealth.}
        \choice unit of account.
        \choice store of value.
    \end{choices}
\end{question}

\begin{question}
    To increase the money supply, the central bank:
    \begin{choices}
        \choice \textbf{buys government bonds.}
        \choice sells government bonds.
        \choice buys corporate stocks.
        \choice sells corporate stocks.
    \end{choices}
\end{question}

\begin{explanation}
    The money supply is the quantity of money available in the economy.
    $$M = C + D = m \cdot B = m \cdot (C + R)$$
    Where
    \begin{itemize}
        \item $C$ = currency = physical money in circulation.
        \item $D$ = demand deposits = bank accounts.
        \item $m$ = money multiplier.
        \item $B$ = base money = reserves + currency in circulation.
    \end{itemize}

    To increase the money supply, the central bank uses open-market operations,
    specifically the purchase of government bonds.

    The mechanism works as follows:
    \begin{enumerate}
        \item The central bank buys government bonds from the public or banks.
        \item To pay for these bonds, the central bank creates new money (increasing $C$ or $R$).
        \item This action directly increases the Monetary Base ($B$).
        \item Because of the Money Multiplier ($m$), where $m > 1$ in a fractional-reserve banking system, the increase in the monetary base results in a larger proportional increase in the total money supply ($M$).
    \end{enumerate}

    Conversely, if the central bank wanted to \textit{decrease} the money supply, it would \textit{sell} government bonds, thereby removing money from the monetary base.
\end{explanation}

\begin{question}
    The money supply will decrease if the:
    \begin{choices}
        \choice monetary base increases.
        \choice \textbf{currency-deposit ratio increases.}
        \choice discount rate decreases.
        \choice reserve-deposit ratio decreases.
    \end{choices}
\end{question}

\begin{explanation}
    The money supply is determined by the equation $M = m \times B$,
    where $m$ is the money multiplier:
    $$m = \frac{cr + 1}{cr + rr}$$

    \textbf{The Currency-Deposit Ratio ($cr$):}
    The $cr$ reflects the preferences of households and firms regarding the form of money they wish to hold (physical currency $C$ vs. bank deposits $D$).

    If the currency-deposit ratio increases, households are deciding to hold relatively more
    cash in hand and less in bank accounts. This action removes reserves from the banking system,
    reducing the banking system's ability to create money.

    Mathematically, because the reserve-deposit ratio ($rr$) is less than 1
    (e.g., 0.1 or 10\%)$$rr < 1$$an increase in $cr$ increases the denominator ($cr + rr$)
    of the multiplier equation more effectively than it increases the numerator ($cr + 1$).
    Consequently, \textbf{an increase in $cr$ causes the money multiplier $m$ to fall}.
    A lower multiplier results in a decreased total money supply.

    \vspace{0.5em}
    \hrule
    \vspace{0.5em}

    \textbf{1. Reserve-deposit ratio} $rr$:
    \begin{itemize}
        \item It dictates that banks must hold a specific fraction of their deposits as reserves rather than lending them out.
        \item By increasing the reserve requirement, the central bank reduces the amount of money banks can lend ($D$),
              thereby reducing the money multiplier $m=\frac{cr+1}{cr+rr}$ and the money supply $M=m\cdot B$.
              Conversely, decreasing the reserve requirement allows banks to lend more.
    \end{itemize}

    \textbf{2. Open-Market Operations}:
    \begin{itemize}
        \item The central bank buys or sells government bonds in the open market, to the public or private banks.
        \item When the central bank buys bonds, it pays for them with \textbf{newly created money}, increasing the Monetary Base ($B$).
        \item When the central bank sells bonds, it removes money from circulation, decreasing the monetary base and bank reserves.
    \end{itemize}

    \textbf{3. Discount Rate}:
    \begin{itemize}
        \item The discount rate is the interest rate that the central bank charges private banks when they borrow money from it.
        \item A lower discount rate encourages banks to borrow more reserves from the central bank,
              which increases the monetary base and potentially the money supply. A higher rate discourages
              borrowing.
    \end{itemize}

    \textbf{4. Interest on Reserves}:
    \begin{itemize}
        \item The central bank pays interest on the reserves that private banks hold on deposit with the central bank.
        \item By changing this interest rate, the central bank can give banks an incentive to hold more or fewer reserves.
              For instance, a higher interest rate encourages banks to hold onto reserves rather than lending them out.
    \end{itemize}
\end{explanation}

\begin{question}
    If the monetary base equals \$400 billion, the currency-deposit ratio equals 0.5, and the reserve-deposit ratio
    equals 0.1, then the money supply equals:
    \begin{choices}
        \choice \$200 billion.
        \choice \$400 billion.
        \choice \$800 billion.
        \choice \textbf{\$1,000 billion.}
    \end{choices}
\end{question}

\begin{explanation}
    From the question we know that
    $$B = \SI{400}{\billion}, \quad cr = 0.5, \quad rr = 0.1$$
    Now the money supply $M$ equals the monetary base $B$ times the money multiplier $m$:
    $$M = m \cdot B = \frac{cr + 1}{cr + rr} \cdot B = \frac{0.5 + 1}{0.5 + 0.1} \cdot \SI{400}{\billion} = \SI{1000}{\billion}$$
\end{explanation}

\begin{question}
    If the ratio of reserves to deposits ($rr$) increases, while the ratio of currency to deposits ($cr$) is constant and
    the monetary base ($B$) is constant, then:
    \begin{choices}
        \choice it cannot be determined whether the money supply increases or decreases.
        \choice the money supply increases.
        \choice \textbf{the money supply decreases.}
        \choice the money supply does not change.
    \end{choices}
\end{question}

\begin{explanation}
    The money supply $M$ is given by the equation
    $$M = m \cdot B, \qquad m = \frac{cr + 1}{cr + rr}$$
    So if $rr \uparrow, cr = \text{ constant}, B = \text{ constant}$, then $m \downarrow$ and $M \downarrow$.
\end{explanation}

\begin{question}
    If the number of employed workers equals 200 million and the number of unemployed workers equals 20
    million, the unemployment rate equals \blank percent (rounded to the nearest percent).
    \begin{choices}
        \choice 0
        \choice \textbf{9}
        \choice 11
        \choice 20
    \end{choices}
\end{question}

\begin{explanation}
    \begin{align*}
         & \text{\# of employed workers}     &  & = E                                                                                                 &  & = \SI{200}{\million}  \\
         & \text{\# of unemployed workers}   &  & = U                                                                                                 &  & = \SI{20}{\million}   \\
         & \text{\# workers in labour force} &  & = L                                                                                         = E + U &  & = \SI{220}{\million}  \\
         & \text{unemployment rate}          &  & = 100 \cdot \frac{U}{L} \% = 100 \cdot \frac{\SI{20}{\million}}{\SI{220}{\million}}                 &  & = \SI{9.09}{\percent}
    \end{align*}
\end{explanation}

\begin{question}
    If the fraction of employed workers who lose their jobs each month (the rate of job separation) is 0.01 and
    the fraction of the unemployed who find a job each month is 0.09 (the rate of job findings), then the natural rate
    of unemployment is:
    \begin{choices}
        \choice 1 percent.
        \choice 9 percent.
        \choice \textbf{10 percent.}
        \choice about 11 percent.
    \end{choices}
\end{question}

\begin{explanation}
    It is given that
    $$\text{rate of job separation} = s = 0.01, \quad\quad \text{rate of job findings} = f = 0.09$$
    The natural rate of unemployment is given by the equation
    $$\text{natural rate of unemployment} = \frac{s}{s + f} = \frac{0.01}{0.01 + 0.09} = \frac{1}{10} = \SI{10}{\percent}$$
\end{explanation}

\begin{question}
    Frictional unemployment is unemployment caused by:
    \begin{choices}
        \choice wage rigidity.
        \choice minimum-wage legislation.
        \choice \textbf{the time it takes workers to search for a job.}
        \choice clashes between the motives of insiders and outsiders.
    \end{choices}
\end{question}

\begin{explanation}
    We have the following unemployment types:
    \begin{enumerate}
        \item \textbf{Structural Unemployment}: mismatch between skills of workers and requirements of jobs
        \item \textbf{Frictional Unemployment}: time it takes to match workers with jobs
        \item \textbf{Cyclical Unemployment}: caused by economic downturns
    \end{enumerate}
\end{explanation}

\begin{question}
    If there are 100 transactions in a year and the average value of each transaction is \$10, then if there is \$200
    of money in the economy, transactions velocity is \blank
    times per year.
    \begin{choices}
        \choice 0.2
        \choice 2
        \choice \textbf{5}
        \choice 10
    \end{choices}
\end{question}

\begin{explanation}
    The velocity of money is given by
    $$V = \frac{T}{M} = \frac{\text{value of all transactions}}{\text{money supply}}$$
    In this case, we have $T=100 \cdot \$10 = \$1000$ and $M=\$200$, so
    $$V = \frac{\$1000}{\$200} = 5$$
\end{explanation}

\begin{question}
    The quantity theory of money assumes that:
    \begin{choices}
        \choice income is constant.
        \choice \textbf{velocity is constant.}
        \choice prices are constant.
        \choice the money supply is constant.
    \end{choices}
\end{question}

\begin{explanation}
    We assume that velocity $V$ is constant and exogenous: $V = \bar{V}$.
\end{explanation}

\begin{question}
    The opportunity cost of holding money is the:
    \begin{choices}
        \choice \textbf{nominal interest rate.}
        \choice real interest rate.
        \choice federal funds rate.
        \choice prevailing Treasury bill rate.
    \end{choices}
\end{question}

\begin{explanation}
    The nominal interest rate $i$ is the interest rate that the bank pays the holder of the money. \\
    The real interest rate $r$ is the nominal interest rate minus the inflation rate: $r = i - \pi$.

    The correct answer is the \textbf{nominal interest rate} ($i$).
    To understand why, we must distinguish between the decision to hold
    liquid cash versus the decision to invest in physical capital.

    \textbf{1. Opportunity Cost of Holding Money ($i$)}
    The demand for money is a choice between holding \textit{liquid currency} (which pays $0\%$ interest) and holding \textit{interest-bearing assets} like bonds or savings accounts (which pay the nominal rate $i$).
    \begin{itemize}
        \item By holding cash, you miss the interest payments you would have received from the bank.
        \item While inflation ($\pi$) erodes the value of cash, it also erodes the value of the principal in the bond. Since inflation affects both options equally, it cancels out of the comparison.
        \item The net cost is simply the interest payment you missed: $i$.
    \end{itemize}

    \textbf{2. Opportunity Cost of Investing ($r$)}
    The \textbf{real interest rate} ($r = i - \pi$) is the opportunity cost of using funds to finance \textit{investment spending} (buying physical capital like machines or factories).
    \begin{itemize}
        \item You can, for example, invest in a machine that will generate goods in the future.
        \item The machine produces goods. The price of those goods will likely rise with inflation.
              Because the investment's return is naturally ``protected'' against inflation (since output prices rise),
              you don't need to subtract inflation from the investment's side.
        \item However, if you didn't do the investment, you would lend your money, which would pay
              you the nominal interest rate $i$, but because of inflation, you would only get $r=i-\pi$.
        \item Therefore, investment demand ($I$) depends on $r$, while money demand ($L$) depends on $i$.
    \end{itemize}
\end{explanation}

\begin{question}
    Consider the money demand function that takes the form
    $$(M / P)^d = Y / (4i)$$
    where $M$ is the quantity of money, $P$ is the price level,
    $Y$ is real output, and $i$ is the nominal interest rate.
    What is the average velocity of money in this economy?
    \begin{choices}
        \choice i
        \choice 4
        \choice 1 / (4i)
        \choice 0.25
    \end{choices}
\end{question}

\begin{explanation}
    The \textit{real} money demand $(M/P)^d$ depends negatively on $i$, the nominal interest rate the opportunity cost of holding money,
    and positively on $Y$, the \textit{real} GDP:
    $$(M/P)^d = L(i, Y) = Y / (4i)$$
    The velocity of money $V$ is defined by the Quantity Equation:
    $$M \cdot V = P \cdot Y$$
    We can rearrange this equation to solve for velocity $V$:
    $$V = \frac{P \cdot Y}{M} = \frac{Y}{M/P}$$
    In equilibrium, the supply of real money balances $M/P$ must equal the demand for real money balances $(M/P)^d$. Therefore, we can substitute the given money demand function into the denominator:
    $$V = \frac{Y}{(Y / 4i)}$$
    Simplifying the expression:
    $$V = Y \cdot \frac{4i}{Y} = 4i$$
    Thus, the velocity of money in this economy is $4i$. This makes sense intuitively:
    as the nominal interest rate $i$ increases, the opportunity cost of holding money rises.
    People hold less money (lower $M/P$), so the money that remains in circulation must change hands
    more frequently (higher $V$) to support the same level of transactions $Y$.
\end{explanation}

\begin{question}
    Variables expressed in terms of physical units or quantities are called
    variables.
    \begin{choices}
        \choice \textbf{real}
        \choice nominal
        \choice endogenous
        \choice exogenous
    \end{choices}
\end{question}

\begin{explanation}
    \textbf{Real variables}:
    \begin{itemize}
        \item Measured in physical units or relative prices.
        \item E.g. $Y$ = real GDP, real wage, real interest rate $r$, ...
    \end{itemize}
    \textbf{Nominal variables}:
    \begin{itemize}
        \item Measured in money units.
        \item E.g. $P$ = number of dollars needed for a representative basket of goods, nominal wage in dollars, nominal interest rate $i$ in dollars, ...
    \end{itemize}
\end{explanation}

\begin{question}
    The theoretical separation of real and monetary variables is called:
    \begin{choices}
        \choice \textbf{the classical dichotomy.}
        \choice monetary neutrality.
        \choice the Fisher effect.
        \choice the quantity theory of money.
    \end{choices}
\end{question}

\begin{question}
    The value of net exports is also the value of:
    \begin{choices}
        \choice net investment.
        \choice net saving.
        \choice national saving.
        \choice \textbf{the difference of national saving and domestic investment.}
    \end{choices}
\end{question}

\begin{explanation}
    $$\underbrace{NX = X - IM}_{\text{net exports}} = \underbrace{NCO = S-I}_{\text{net capital outflow}}$$
\end{explanation}

\begin{question}
    In a small open economy, if domestic investment exceeds domestic saving, then the extra investment will be financed by:
    \begin{choices}
        \choice \textbf{borrowing from abroad.}
        \choice borrowing from domestic banks.
        \choice the domestic government.
        \choice the World Bank.
    \end{choices}
\end{question}

\begin{explanation}
    It is given that
    $$\text{domestic investment} = I > S = \text{domestic saving}$$
    This means that
    $$\text{net capital outflow} = NCO = S - I < 0$$
    This means that the country is a net borrower. Of course, it has to borrow from \textit{abroad}.

    Why can't it borrow from the World Bank? Wealthy, developed nations like Denmark often run trade deficits or surpluses.
    When they run a deficit ($I>S$), they borrow from international private markets (Wall Street, London, etc.).
    They do not borrow from the World Bank. Therefore, ``borrowing from the World Bank'' is not a valid general
    answer, because it only applies to specific developing nations under specific conditions, whereas
    ``borrowing from abroad'' applies to any open economy where $I>S$.
\end{explanation}

\begin{question}
    If a graph $NX(\varepsilon)^{-1}$ is drawn with net exports $NX$ on the horizontal axis and the \textit{real} exchange rate $\varepsilon$ on the vertical axis,
    then the real exchange rate is determined by the intersection of the \blank
    net-exports schedule and the \blank line representing saving minus investment.
    \begin{choices}
        \choice \textbf{downward-sloping; vertical}
        \choice upward-sloping; vertical
        \choice downward-sloping; upward-sloping
        \choice upward-sloping; downward-sloping
    \end{choices}
\end{question}

\begin{explanation}
    The real exchange rate $\varepsilon$ represents the value of one country's output relative to another country's output.\\
    If $\varepsilon$ is high, domestic goods become more expensive relative to foreign goods, so net exports fall and imports rise.\\
    If $\varepsilon$ is low, foreign goods become more expensive relative to domestic goods, so net exports rise and imports fall.

    The net-exports function $NX(\varepsilon)$ reflects this \textbf{inverse} relationship between net exports $NX$ and the real exchange rate $\varepsilon$.
    $\implies \textbf{downward-sloping}$

    The accounting identity says that $NX = X - IM = NCO = S - I$. Here, $S$ depends on domestic factors ($\bar{S} = \bar{Y} - C(\bar{Y} - \bar{T}) - \bar{G}$) and $I$ is determined by the world interest rate $r^*$,
    so we have that the real exchange rate $\varepsilon$ adjusts to balance the accounting identity:
    $$NX(\varepsilon) = \bar{S} - I(r^*)$$
    Since $\bar{S}$ and $I(r^*)$ don't depend on $\varepsilon$, we have that $\bar{S} - I(r^*)$ is a \textbf{vertical line.}
\end{explanation}

\begin{question}
    In a small open economy, if the world interest rate $r^*$ falls, then domestic investment will \blank, and the real
    exchange rate will \blank, holding all else constant.
    \begin{choices}
        \choice decrease; decrease
        \choice decrease; increase
        \choice increase; decrease
        \choice \textbf{increase; increase}
    \end{choices}
\end{question}

\begin{explanation}
    Investment $I$ depends \textbf{negatively} on the world interest rate $r^*$:
    $$r^* \downarrow \implies \boxed{I(r^*) \uparrow}$$
    This decreases $NCO = S - I(r^*) = NX = X - IM$. Because $NX(\varepsilon)$ is downward-sloping, this means that $\varepsilon$ must \textbf{increase}.
\end{explanation}

\begin{question}
    A depreciation of the real exchange rate in a small open economy could be the result of:
    \begin{choices}
        \choice a domestic tax cut.
        \choice an increase in government spending.
        \choice a decrease in the world interest rate.
        \choice \textbf{the expiration of an investment tax-credit provision.}
    \end{choices}
\end{question}

\begin{explanation}
    \begin{itemize}
        \item \textbf{Choice D (Correct):}\\ The expiration of an investment tax credit reduces investment demand ($I \downarrow$).
              $$I \downarrow \implies NCO = S - I \uparrow \implies NX(\varepsilon) \uparrow \implies \boxed{\varepsilon \downarrow}$$

        \item \textbf{Choice A:}\\ A tax cut ($T \downarrow$) increases disposable income ($Y-T \uparrow$) and consumption ($C \uparrow$).
              $$C \uparrow \implies S = Y - C - G \downarrow \implies NCO \downarrow \implies NX(\varepsilon) \downarrow \implies \varepsilon \uparrow$$

        \item \textbf{Choice B:}\\ An increase in government spending ($G \uparrow$) reduces national saving.
              $$G \uparrow \implies S \downarrow \implies NCO \downarrow \implies \varepsilon \uparrow$$

        \item \textbf{Choice C:}\\ A lower world interest rate ($r^* \downarrow$) stimulates investment.
              $$r^* \downarrow \implies I \uparrow \implies NCO \downarrow \implies \varepsilon \uparrow$$
    \end{itemize}
\end{explanation}

\begin{question}
    \begin{center}
        \includegraphics[width=0.5\textwidth]{images/policies1.jpg}
    \end{center}

    Which of the above panels illustrates the impact of contractionary fiscal policies at home on the real exchange rate?
    \begin{choices}
        \choice $\leftarrow$
        \choice
        \choice
        \choice
    \end{choices}
\end{question}

\begin{question}
    \begin{center}
        \includegraphics[width=0.5\textwidth]{images/policies2.jpg}
    \end{center}

    Which of the above panels illustrates
    the impact of an increase in household saving on the real exchange rate?
    \begin{choices}
        \choice $\leftarrow$
        \choice
        \choice
        \choice
    \end{choices}
\end{question}

\begin{explanation}
    Increase household saving means that the private saving goes up:
    $$(Y - T) - C \uparrow \implies S = Y - C - G \uparrow \implies
        NX = S - I \uparrow$$
\end{explanation}

\begin{question}
    The percentage change in the nominal exchange rate equals the percentage change in the real exchange rate
    plus the:
    \begin{choices}
        \choice \textbf{foreign inflation rate minus the domestic inflation rate.}
        \choice domestic inflation rate minus the foreign inflation rate.
        \choice foreign exchange rate minus the domestic exchange rate.
        \choice domestic interest rate minus the foreign interest rate.
    \end{choices}
\end{question}

\begin{explanation}
    Because the nominal exchange rate $e$ is equal to $e=\varepsilon \cdot \frac{P^*}{P}$, we have that:
    $$\frac{\Delta e}{e} = \frac{\Delta \varepsilon}{\varepsilon} + \left(\frac{\Delta P^*}{P^*} - \frac{\Delta P}{P}\right) = \frac{\Delta \varepsilon}{\varepsilon} + \pi^* - \pi
        \implies \boxed{\% \Delta e = \% \Delta \varepsilon + \underbrace{\% \Delta \pi^* - \% \Delta \pi}_{\text{foreign - domestic}}}$$
\end{explanation}

\begin{question}
    Holding other factors constant, legislation to cut taxes in an open economy will:
    \begin{choices}
        \choice increase national saving and lead to a trade surplus.
        \choice increase national saving and lead to a trade deficit.
        \choice reduce national saving and lead to a trade surplus.
        \choice \textbf{reduce national saving and lead to a trade deficit.}
    \end{choices}
\end{question}

\begin{explanation}
    Cutting taxes:
    $$T \downarrow \implies \underbrace{Y-T \uparrow}_{\text{disposable income}} \implies \underbrace{C = C(Y-T)}_{\text{consumption}} \uparrow \implies \underbrace{S = Y - C - G}_{\text{national saving}} \downarrow \implies \underbrace{NX = S - I}_{\text{net exports}} \downarrow$$
    So national savings decrease and $NX$ decreases. When $NX < 0$, there is a trade deficit.
\end{explanation}

\end{document}
