\documentclass[dtu]{dtuarticle}
\usepackage{parskip} % use enters instead of indents

\usepackage{amsmath}
\usepackage{amssymb}
\usepackage{bm}
\usepackage{url}
\usepackage{hyperref}
\usepackage{subcaption}
\usepackage{siunitx}
\usepackage[most]{tcolorbox} % most is required for breakable
\usepackage{listings}
\usepackage{color}

\newcommand{\N}{\mathbb{N}}
\newcommand{\Z}{\mathbb{Z}}
\newcommand{\Zn}{\mathbb{Z}_n}
\newcommand{\Zp}{\mathbb{Z}_{\geq 0}}
\newcommand{\Zpp}{\mathbb{Z}_{>0}}
\newcommand{\Q}{\mathbb{Q}}
\newcommand{\R}{\mathbb{R}}
\newcommand{\C}{\mathbb{C}}
\newcommand{\F}{\mathbb{F}}
\newcommand{\E}{\mathbb{E}}
\newcommand{\eqDef}{\overset{\Delta}{=}}
\newcommand{\plusn}{+_n}
\newcommand{\timesn}{\cdot_n}
\newcommand{\id}{\mathrm{id}}
\newcommand{\gen}[1]{\langle #1 \rangle} % subgroup generated by #1
\newcommand{\red}[1]{\color{red}#1\color{black}}
\newcommand{\ord}{\operatorname{ord}}
\newcommand{\lcm}{\operatorname{lcm}}
\newcommand{\im}{\operatorname{im}}

\newcounter{qnumber}[section]
% \renewcommand{\theqnumber}{\thesubsection.\arabic{qnumber}}
\newenvironment{question}[1][]{
    \def\qpoints{#1}
    \refstepcounter{qnumber}
    \par\medskip\noindent    % Starts new para, adds space, removes indent
    \textbf{Question \theqnumber:}
}{
    % Check if the argument is empty
    \ifx\qpoints\empty
    \else
        \textbf{(\qpoints\ points)}
    \fi
    \par\medskip % Adds space after the question
}

\newtcolorbox{answer}{
    breakable,             % Allows splitting across pages
    colback=white,         % Background color
    colframe=gray,        % Border color
    % only top rule
    boxrule=0mm,
    toprule=0.2mm,
    width=\dimexpr\textwidth\relax, % Set width
    arc=0pt, outer arc=0pt,% Makes corners sharp (like tabular)
    left=0.0cm, right=0.0cm,   % Padding inside the box
    top=0.2cm, bottom=0.0cm,   % Padding inside the box
    parbox=false,          % Uses standard paragraph mode (better spacing)
    before={\textcolor{gray}{\sffamily\bfseries\footnotesize ANSWER}\vspace{0.1cm}}
}

\title{Exam Questions}
\subtitle{Discrete Mathematics 2: Algebra}
\author{Vincent Van Schependom}
\course{010158 Discrete Mathematics 2: Algebra}
\address{
	DTU Compute \\
	Fall 2025
}
\date{Fall 2025}



\begin{document}

\maketitle

\section*{Introduction}

I created this document based on previous exams of the course Discrete Mathematics 2: Algebra.
It contains \textit{fully complete} answers to frequently (or always) asked (sub)questions in the exam.
The answers are derived from my own lecture notes, exercises and the course material \cite{course_notes}.

Per question, I tried collecting every single type of subquestion that might be asked.
Note that this may lead to repetitive and/or unrelated answers within each question.
On the actual exam there are less subquestions, of course.
Also note that Question 2 about groups differs from year to year, so it is rather hard to predict what will be asked.
Hence, this question does not contain many subquestions, nor answers.

Finally note that this document was created in the Fall of 2025; course materials may evolve in future semesters.
Also, the answers are not guaranteed to be correct.
If you identify any errors, please contact the author, Vincent Van Schependom.

\bibliography{references}
\bibliographystyle{unsrt}

\newpage

\begin{question}
    Let $(S_5, \circ)$ be the group of permutations of $A = \{1, 2, 3, 4, 5\}$. Let $f$ denote the permutation
    $$f := \begin{pmatrix}
            1 & 2 & 3 & 4 & 5 \\
            2 & 3 & 4 & 5 & 1
        \end{pmatrix}$$
    \begin{enumerate}
        \renewcommand{\labelenumi}{(\alph{enumi})}
        \item Write $f$ as a composition of disjoint cycles.
        \item What are the order and the cycle type of $f$?
        \item What is the smallest natural number $n$ such that $S_n$ contains a permutation
              of order 10? Motivate your answer.
        \item Does $S_9$ contain a permutation of order 18? Motivate your answer.
        \item What is the maximal order a permutation of $S_6$ can have? Motivate your answer.
    \end{enumerate}
\end{question}

\begin{answer}
    \begin{enumerate}
        \renewcommand{\labelenumi}{(\alph{enumi})}
        \setlength{\itemsep}{2em}
        \item The disjoint cycle decomposition of $f$ is$$f = c_1 \circ \ldots \circ c_k$$ where $c_i$ are disjoint cycles and $\ord(c_i) = \ell_i=$ number of elements in $c_i$ .
        \item We're looking for the smallest integer $i \in \Z_{>0}$ such that $f^i = \id_A$.

              Because the disjoint cycle decomposition of $f$ consists of $k$ cycles $c_i$ of length $\ell_i$,
              it follows from Proposition 2.3.12 that $$\ord(f) = \lcm(\ell_1, \ldots, \ell_k)$$
              To compute the cycle type, let $t_1$ be the number of elements in $A$ that are fixed by $f$
              and for $i>1$, let $t_i$ be the number of $t_i$-cycles in the DCD of $f$. Then the cycle type
              of $f$ is $$(t_1, \ldots, t_{\red{n}})$$

        \item If $n = 7$ then a permutation $g \in S_n$ of order 10 can be found, for example $$g = (1\, 2)(3\, 4\, 5\, 6\, 7)$$
              Now we show that such a permutation does not exist if $n \leq 6$, which proves that
              $n = 7$ is the smallest natural number such that $S_n$ contains a permutation of order 10.

              Assume that $n \leq 6$. Assume further that, by contradiction, $g \in S_n$ of order
              $\ord(g) = 10$ exists. We know that, if $g = c_1 \circ c_2 \circ \ldots \circ c_k$
              is the disjoint cycles decomposition of $g$, and $c_i$ is a cycle of length
              $\ell_i$ for $i= 1, \ldots, k$, then $$\ord(g) = \lcm(\ell_1, \ldots, \ell_k)$$

              This implies that every cycle length $\ell_i$ must divide 10.
              Thus, $\ell_i \in \{1,2,5,10\}$. Since $n \leq 6$,
              a 10-cycle is impossible, so the lengths must be 1, 2, or 5.
              The cycles $c_i$ thus need to be 5-cycles, 2-cycles or 1-cycles.

              Now it holds that
              $$n = 1 \cdot t_1 + 2 \cdot t_2 + \ldots + n \cdot t_n = \red{1 \cdot t_1} + 2 \cdot t_2 + 5 \cdot t_5 \red{\geq} 2 \cdot t_2 + 5 \cdot t_5$$
              Since $t_1 \geq 0$ and at least one of \textit{each} needs to appear in the decomposition of
              $f$ (i.e. $t_2 \geq 1$ and $t_5 \geq 1$), it follows that $n \geq 2 \cdot 1 + 5 \cdot 1 = 7$, a \textit{contradiction}.

        \item No. Indeed, if $f \in S_9$ and we write $f = c_1 \circ \ldots \circ c_k$ as a disjoint cycle decomposition where $c_i$ is a cycle of length $\ell_i$, then $\ord(f) = \lcm(\ell_1, \ldots, \ell_k)$.

              To have $\ord(f) = 18$, each $\ell_i$ must be a divisor of 18 and satisfy $\ell_i \leq 9$. Thus, the possible cycle lengths are:
              $$ \ell_i \in \{1, 2, 3, 6, 9\} $$
              We distinguish two cases:
              \begin{itemize}
                  \item \textbf{Case 1: No cycle of length 9 exists.} \\
                        In this case, all $\ell_i \in \{1, 2, 3, 6\}$. None of these numbers are divisible by 9 (they are either not divisible by 3, or divisible by 3 but not 9). Consequently, their least common multiple cannot be divisible by 9. Since $18$ is divisible by 9, $\lcm(\ell_1, \ldots, \ell_k) \neq 18$.

                  \item \textbf{Case 2: A cycle of length 9 exists.} \\
                        If there is a cycle of length $\ell_j = 9$, then because the cycles are disjoint and the total number of elements is 9 (i.e., $\sum \ell_i \leq 9$), no other non-trivial cycles can exist in the decomposition. Thus $f$ is a 9-cycle, which implies $\ord(f) = 9 \neq 18$.
              \end{itemize}
              Since both cases fail to produce an order of 18, such an element cannot exist in $S_9$.

              \newpage
        \item Let $f \in S_6$ and write $f = c_1 \circ c_2 \circ \ldots \circ c_k$
              as a disjoint cycles decomposition where $c_i$ is a cycle of length
              $\ell_i$ for $i= 1, \ldots, k$. To compute $\ord(f) = \lcm(\ell_1, \ldots, \ell_k)$
              we need to understand how the different cycle lengths $\ell_i$ can be.
              To do so, let $m$ denote the maximum of the lengths $\ell_i$ of the cycles $c_i$ with
              $i= 1, \ldots, k$. Clearly $m \leq 6$. We divide some cases:
              \begin{itemize}
                  \item $m = 6$. Then $k= 1$ and $f$ is a 6-cycle. The order of $f$ is 6 in this case.
                  \item $m = 5$. Then $f$ is a 5-cycle (recall that any 1-cycle is just the identity
                        permutation). Hence the order of $f$ is 5 in this case.
                  \item $m = 4$. Hence either $f$ is a 4-cycle, or the composition of a 4-cycle and
                        a 2-cycle. In any case the order of $f$ is 4 as $\lcm(4, 2) = 4$.
                  \item $m = 3$. Here either $f$ is a 3-cycle, or the composition of 2 3-cycles or
                        the composition of a 3-cycle and a 2-cycle. In the first 2 cases the order
                        of $f$ is 3 while in the last case $\ord( f ) = \lcm(3, 2) = 6$.
                  \item $m = 2$. Then $f$ is composition of 2-cycles and hence its order is 2.
              \end{itemize}
              We conclude that the order of $f$ is at most 6.
    \end{enumerate}
\end{answer}

\newpage
Note that for any $m$-cycle $g = (a_0 \, a_1 \, \ldots \, a_{m-1})$ of order $\ord(g) = m$, we have that $$g^m = \id$$
and furthermore, we have that
$$g^i = g^{i \bmod m}$$
The above obviously holds if $0 \leq i < m$, as in this case $(i \bmod m) = i$.\\
In the general case, we can perform \textit{division with remainder}
to write $$i = qm + r, \qquad \text{with } 0 \leq r = (i \bmod m) < m \text{ and } q \text{ possibly negative}$$
and then we have that
$$g^i = g^{qm + r} = (g^m)^q \cdot g^r = \id^q \cdot g^r = g^r,$$
and the result follows, as $r = i \bmod m$.

\vspace{2cm}

\begin{question}
    Consider the group $(G, \cdot)$ and consider the subgroup $H := \ldots$.
    \begin{enumerate}
        \renewcommand{\labelenumi}{(\alph{enumi})}
        \item Show that $H$ is a subgroup of $G$.
        \item Show that $\varphi : \begin{cases} G \to V \\ g \mapsto \ldots \end{cases}$
              is a group homomorphism.
        \item Determine whether $(G_1, \cdot_1)$ is isomorphic to $(G_2, \cdot_2)$.
    \end{enumerate}
\end{question}

\begin{answer}
    \begin{enumerate}
        \renewcommand{\labelenumi}{(\alph{enumi})}
        \setlength{\itemsep}{2em}
        \item The identity element $e_G$ is in $H$: ...

              We can now use Lemma 4.1.2 (proven in an exercise), which says that any \textit{non-empty} set $H \subseteq G$ is a subgroup of $G$ if and only if
              $$\forall f, g \in H : f^{-1} \cdot g \in H$$

        \item We prove that the two axioms of a group homomorphism (Definition 6.1.1) hold:
              \begin{enumerate}
                  \item $\varphi(e_G) = e_V$: ...
                  \item $\varphi(f \cdot_G g) = \varphi(f) \cdot_V \varphi(g)$: ...
              \end{enumerate}

              \newpage

        \item In order to be isomorphic, the cardinalities must match: $|G_1| = |G_2|$.

              Furthermore, all elements in $G_1$ must have the same order as the corresponding elements in $G_2$
              in case both groups are isomorphic.

              To see this, assume that the groups are isomorphic ($G_1 \simeq G_2$) via the isomorphism $\psi : G_1 \to G_2$.\\
              Let $\ord(g) = n$ for $g \in G_1$.
              $$\psi(g)^n = \underbrace{\psi(g) \cdot_{2} \ldots \cdot_{2} \psi(g)}_n \stackrel{(2)}{=} \psi(g \cdot_{1} \ldots \cdot_{1} g) = \psi(g^{n}) \stackrel{\text{ord}(g)=n}{=} \psi(e_{1}) \stackrel{(1)}{=} e_{2}.$$
              We conclude that $\psi(g)^n = e_{2}$, so because of Lemma 3.1.12, $\ord(\psi(g))$ divides $n$.
              Now we show that $n$ is the smallest such positive integer.

              Assume there exists a positive integer $m < n$ such that $\psi(g)^m = e_{2}$.
              Then, by similar reasoning as above, we have $\psi(g^{m}) = e_{2}$.

              Now we see that $$\psi \text{ is bijective} \quad \implies \quad \psi \text{ is injective } \quad \stackrel{\text{Lemma 6.1.8}}{\iff} \quad \ker(\psi) = \{ e_{1} \} \quad \implies \quad g^{m} = e_{1}$$But we assumed that $\ord(g) = n$ and that $m<n$. This gives a contradiction, since the order is by definition the smallest positive integer such that $g^{\ord(g)} = e_{1}$.
              Therefore, no such $m$ can exist and we conclude that $\ord(\psi(g)) = n$.

    \end{enumerate}
\end{answer}

\newpage

\begin{question}
    Consider the set $R = \ldots$.
    \begin{enumerate}
        \renewcommand{\labelenumi}{(\alph{enumi})}
        \item Show that $(R, +,\cdot)$ is a ring.
        \item Let $I := \ldots \subseteq R$. Prove that $I$ is an ideal of $R$.
        \item Determine whether or not $I=R$.
        \item Consider the map $\varphi : R \to S$. Show that $\varphi$ is a ring homomorphism.
        \item Compute the kernel and image of $\varphi$.
        \item Prove that the quotient ring $R/I$ is isomorphic to $S$.
    \end{enumerate}
\end{question}

\begin{answer}
    \begin{enumerate}
        \renewcommand{\labelenumi}{(\alph{enumi})}
        \setlength{\itemsep}{2em}
        \item We prove that $(R, +,\cdot)$ is a ring by showing that it satisfies all the ring
              axioms from Definition 7.1.1. The zero-element is $0_R = ...$ and the one-element is $1_R = ...$.
              \begin{enumerate}
                  \renewcommand{\labelenumii}{(\arabic{enumii})}
                  \item $(R, +)$ is an abelian group.
                        \begin{enumerate}
                            \renewcommand{\labelenumiii}{(\roman{enumiii})}
                            \item There exists an identity element $0_R \in R$.
                            \item The operation $+$ is \textbf{associative}. Take $a,b,c \in R$ arbitrarily.\\
                                  (\textit{Associativity holds in the larger
                                      group $G$: for any $a', b', c' \in R$ we have $a' \cdot (b' \cdot c') = (a' \cdot b') \cdot c'$,
                                      so this particularly holds for $a = a', b = b', c = c'$})
                            \item $(R, +)$ is \textbf{closed under inverses}. Take $a \in R$ arbitrarily.\\
                                  Then, its additive inverse is $a^{-1}$ is also in $R$.
                            \item $(R, +)$ is \textbf{closed under addition}. Take $r, s \in R$ arbitrarily.
                                  Then, $r + s = \ldots \in R$.
                        \end{enumerate}
                  \item There exists an identity element $1_R \in R$ for the operation $\cdot$:
                        $$\forall f, g \in R : f \cdot 1_R = f = 1_R \cdot g$$
                  \item The operation $\cdot$ is \textbf{associative}.
                  \item The operations $+$ and $\cdot$ satisfy the \textbf{distributive laws}.
              \end{enumerate}

        \item To prove that $I$ is an ideal of $R$, we prove that (1) $I$ is a subgroup of $(R, +)$ and (2) $\forall r \in R, \forall x \in I : rx \in I$.
              \begin{enumerate}
                  \renewcommand{\labelenumii}{(\arabic{enumii})}
                  \item $I$ is a subgroup of $(R, +)$: $$\ldots$$
                  \item Take $r \in R$ and $x \in I$ arbitrarily. Then, $rx = \ldots \in I$.
              \end{enumerate}
              Because both conditions from Definition 8.1.7 are satisfied,
              we conclude that $I$ is an ideal of $R$.

        \item We now show that $I=R$.

              Recall that $I = R$ if and only if $I$ contains the one-element $1_R$:
              $$I = R \iff 1_R \in I$$
              (\textit{Also recall that if a unit $u \in R^*$ is in $I$, then $1_R \in I$ and thus $I = R$:
                  $u \in I \Rightarrow 1_R \in I \Leftrightarrow I = R$})

              Our aim is thus to understand if $I$ contains $1_R$ (or if $I$ contains a unit $u \in R^*$).

        \item To show that $\varphi$ is a ring homomorphism, we show that it satisfies all
              conditions in Definition Definition 8.1.1:
              \begin{enumerate}
                  \renewcommand{\labelenumii}{(\arabic{enumii})}
                  \item $\varphi$ is a group homomorphism:
                        \begin{enumerate}
                            \renewcommand{\labelenumiii}{(\roman{enumiii})}
                            \item $\varphi(0_R) = 0_S$
                            \item $\varphi(r +_R s) = \varphi(r) +_S \varphi(s)$ for all $r, s \in R$.
                        \end{enumerate}
                  \item $\varphi(1_R) = 1_S$
                  \item $\varphi(r \cdot_R s) = \varphi(r) \cdot_S \varphi(s)$ for all $r, s \in R$.
              \end{enumerate}

        \item We compute that \begin{align*}
                  \ker(\varphi) & \eqDef \{ r \in R \mid \varphi(r) = 0_S \} \\
                                & = \ldots                                   \\
                  \im(\varphi)  & \eqDef \{ \varphi(r) \mid r \in R \}       \\
                                & = \ldots
              \end{align*}

        \item We show that $I = \ker(\varphi)$ by proving both inclusions.
              \begin{itemize}
                  \item[$\boxed{\subseteq}$]: Take $x \in I$ arbitrarily. Then, $\varphi(x) = \ldots = 0_S$.
                  \item[$\boxed{\supseteq}$]: Take $x \in \ker(\varphi)$ arbitrarily. Then, $x \in I$.
              \end{itemize}
              Now we show that $\im(\varphi) = S$, i.e. $\varphi$ is surjective. Take any $s \in S$ arbitrarily.
              Then, we can find a preimage $r \in R$ such that $\varphi(r) = s$. This shows that $\im(\varphi) = S$.

              We now apply the \textit{Isomorphism Theorem for Rings} (Theorem 8.3.5), which states that
              $$\bar{\varphi} : \begin{cases}
                      R/\ker(\varphi)   & \to \im(\varphi)   \\
                      r + \ker(\varphi) & \mapsto \varphi(r)
                  \end{cases}$$
              is a ring isomorphism. Because $\ker(\varphi) = I$ and $\im(\varphi) = S$, this
              proves that $$(R/I, +, \cdot) \cong (S, +_S, \cdot_S)$$.
    \end{enumerate}
\end{answer}

\newpage

\begin{question}
    As usual, the finite field with $5$ elements is denoted by $(\F_5, +,·)$, while $(\F_5[X], +,·)$
    denotes the ring of polynomials with coefficients in $\F_5$. Define the quotient ring
    $(R, +, \cdot)$, where $$R := \F_5[X]/\gen{X^4 + 2X^3 + X + 2}$$
    \begin{enumerate}
        \renewcommand{\labelenumi}{(\alph{enumi})}
        \item Compute the standard form of the coset $X^7 + X^6 + 2X^5 + X^4 + 2 + \gen{X^4 +
                      2X^3 + X + 2}$.
        \item Write the polynomial $X^4 +2X^3 + X + 2 \in \F_5[X]$ as the product of irreducible
              polynomials.
        \item Find $z$ distinct zero-divisors in $R$.\\
              \textit{(Only possible if $f(X)$ is \textbf{not} irreducible, otherwise $R$ is a field and thus a domain!)}
        \item Show that $X + 3 + \gen{X^4 + 2X^3 + X + 2}$ is a unit of $R$ and compute its
              multiplicative inverse.
        \item Which are the primitive elements in $\F_5[X]/\gen{X^4 + 2X^3 + X + 2}$?
        \item Does $R$ contain zero divisors? Motivate your answer.
        \item Determine how many units $R$ contains.
        \item Compute the multiplicative order of the element $\alpha := X + \gen{X^2 + X + 1}$ in
              the quotient ring $(\F_5[X]/\gen{X^2 + X + 1}, +,\cdot)$.
    \end{enumerate}
\end{question}

\begin{answer}
    \begin{enumerate}
        \renewcommand{\labelenumi}{(\alph{enumi})}
        \setlength{\itemsep}{2em}
        \item Because $p$ is a prime, $\Z_p = \F_p$ is a \textit{field} and we can apply
              Lemma 8.2.6, which says that any coset $g(X) + \gen{f(X)}$ of the ideal $I := \gen{f(X)}$
              can be \textit{uniquely} described in the standard form
              $$r(X) + \gen{f(X)} \qquad \text{where either } \begin{cases}
                      r(X) = 0  \Leftrightarrow \deg(r(X)) = -\infty \\
                      0 \leq \deg(r(X)) < \deg(f(X))
                  \end{cases}$$
              where $r(X) \in \F_p[X]$ is the \textit{unique} remainder of long
              polynomial division of $g(X)$ by $f(X)$:
              $$g(X) = q(X) \cdot f(X) + r(X)$$

        \item Denote the polynomial that generates the ideal $I := \gen{f(X)}$ as
              $$f(X) := X^4 +2X^3 + X + 2$$
              %   Because $\deg(f(X)) > 3$, not having any roots in $\F_p$ is \textit{not} enough to
              %   conclude that $f(X)$ is irreducible. 
              Whenever $a \in \F_5$ is a root of $f(X)$, then $(X-a) \in \F_p[X]$ divides $f(X)$, providing a proper
              factor, as Proposition 7.4.3 states that in this case
              $f(X)$ can be factored as $$f(X) = (X - a) \cdot q(X)$$
              with $\deg(q(X)) = \deg(f(X)) - 1 = 3$. Because furthermore $(X-a)$ has degree 1,
              Lemma 9.2.4 says that it is our first \textit{irreducible} factor of $f(X)$, which we will denote as
              $h_1(X) := X - a$. We check all $a \in \F_p$ to see if $f(a) \equiv 0 \pmod{p}$.
              Because $-1 \equiv 4 \pmod{5}$ and $-2 \equiv 3 \pmod{5}$, we have to check if
              $f(0), f(1), f(2), f(-1), f(-2) \equiv 0 \pmod{5}$ to check if
              $f(X)$ has any roots.
              $$\ldots$$
              We see that $a \in \F_5$ is a root of $f(X)$. We now compute the factor $q(X)$ from above
              by performing long polynomial division of $f(X)$ by the factor $h_1 := (X - a)$.
              $$\ldots$$
              We now apply the same principle to $q(X)$ to find the remaining factors of $f(X)$.
              $$\ldots$$
              Because $\deg(h_k(X)) \in \{2,3\}$, and $h_k(X)$ has no roots in $\F_5$, it follows from
              Lemma 9.2.5 that $h_k(X)$ is irreducible. We conclude that the factorization of $f(X)$ into
              irreducible polynomials is
              $$f(X) = h_1(X) \cdot \ldots \cdot h_k(X)$$

              \newpage
        \item To find $k$ distinct zero-divisors in $R$, we first note that \textit{any proper monic factor}
              $h(X)$ of $f(X)$ leads to a zero-divisor $h(X) + \gen{f(X)}$. In fact, for such polynomial
              \begin{equation}
                  \deg(\gcd[h(X), f(X)]) = \deg(h(X)) \label{deg_gcd}
              \end{equation}
              and since $h(X)$ is a \textit{proper} factor of $f(X)$, it follows that
              \begin{equation*}
                  0 < \deg(h(X)) < \deg(f(X)) \quad \overset{\eqref{deg_gcd}}{\implies} \quad 0 < \deg(\gcd[h(X), f(X)]) < \deg(f(X))
              \end{equation*}
              Now we can apply Proposition 9.1.2 to conclude that $h(X) + \gen{f(X)}$ is a zero-divisor.
              Because in part (a) we found that $$f(X) = h_1(X) \cdot \ldots \cdot h_k(X)$$
              a proper monic factor $h(X) \in \{h_1(X), \ldots, h_k(X)\}$ leads to a zero-divisor.
              We thus already found $k$ distinct zero-divisors in $R$.

              To find more zero-divisors, we note that any multiple $a(X) \cdot h(X)$ of a proper monic factor
              $h(X)$ of $f(X)$ with degree
              \begin{align*}
                  \deg(a(X) \cdot h(X)) & = \deg(a(X)) + \deg(h(X)) &  & (\F_p \text{ is a domain} \Rightarrow \F_p[X] \text{ is a domain}) \\
                                        & < \deg(f(X))
              \end{align*}
              strictly smaller than $\deg(f(X))$ leads to a zero-divisor $(a(X) \cdot h(X)) + \gen{f(X)}$
              in the quotient ring $\F_q = \F_{p^q} = \F_5[X]/\gen{f(X)}$. Why? Because $h(X)$ divides both $f(X)$ and $a(X) \cdot h(X)$,
              it follows that $\deg(\gcd[f(X), a(X) \cdot h(X)]) \geq \deg(h(X))$. Furthermore, since $h(X)$ is a proper factor,
              it holds that $\deg(h(X)) < \deg(f(X))$. We conclude that
              $$0 < \deg(\gcd[f(X), a(X) \cdot h(X)]) < \deg(f(X)),$$
              so again by Proposition 9.1.2, $h(X) \cdot a(X) + \gen{f(X)}$
              is a zero divisor in $\F_p[X]/\gen{f(X)}$.

              We start with the $p-1$ constant multiples of each of the proper monic factors $h_i(X)$ ($i=1,\ldots,k$) of $f(X)$.
              We thereby find $(p-1) \cdot k$ more distinct zero-divisors in $R$ on top of the $k$ zero-divisors we found above.

              Now also consider the non-constant multiples of each of the proper monic factors $h_i(X)$ ($i=1,\ldots,k$) of $f(X)$.

        \item The given coset $X + 3 + \gen{f(X)} =: g(X) + \gen{f(X)}$ is a unit of $R$ if
              and only if $\deg(\gcd[f(X), g(X)]) = 0$ by Proposition 9.1.2. From the proof of this proposition,
              the inverse of this coset is given by $$[g(X) + \gen{f(X)}]^{-1} = s(X) + \gen{f(X)},$$ in this
              case, where $s(X)$ is obtained by performing the \textit{Extended Euclidean algorithm}
              on $f(X)$ and $g(X)$ to obtain
              $$\gcd[f(X), g(X)] = 1 = s(X) \cdot f(X) + r(X) \cdot g(X)$$
              We now apply the Extended Euclidean algorithm to find $s(X)$ and $r(X)$:
              $$\left[\begin{array}{l | rr}
                          f(X) & 1 & 0 \\
                          g(X) & 0 & 1
                      \end{array}\right] \quad \rightarrow \quad \ldots \quad \rightarrow \quad \left[\begin{array}{c | c c}
                          c      & c \cdot s(X) & c \cdot r(X) \\
                          \ldots & \ldots       & \ldots
                      \end{array}\right]$$
              The identity in row 1 is almost the one we are looking for. To ensure uniqueness,
              we now make the GCD monic by multiplying the identity in row 1 by $c^{-1}$.

        \item Denote $\F_q = \F_{p^q} = \F_p[X]/\gen{f(X)}$.
              \textbf{If} $f(X)$ is irreducible, by Theorem 9.3.2, $\F_q$ is a field with $q = p^d$ elements.
              Because of Theorem 9.4.1, we know that this \textit{finite field} $\F_q$ with $q = p^d$ for prime $p$ and $d \geq 1$
              has \textit{at least one primitive element} $\alpha \in \F_q$. In this case, $(\F_q^*, \cdot)$ is a
              \textit{cyclic} group of order $$\ord(\F_q^*) = |\F_q^*| = \ord(\alpha) = q-1$$ generated by a primitive element $\alpha$:
              $$\F_q^* = \{\alpha^0, \alpha^1, \alpha^2, \ldots, \alpha^{q-2}\}$$
              Because of Lagranges Theorem (more specifically Proposition 4.4.4),
              we know that $$\forall g \in \F_q^* : \quad \operatorname{ord}(g) \text{ divides } |\F_q^*| = q-1$$
              The possible orders of elements in $\F_q^*$ are thus the divisors of $q-1$:
              $$D = \{d \in \Z_{>0} \mid d \text{ divides } (q-1)\}$$
              A primitive element is an element $\alpha \in \F_q^*$ with order $\operatorname{ord}(\alpha) = q-1$.
              We thus find a primitive element $\alpha$ by finding an element
              whose order is not any of the other divisors of $q-1$:
              $$\ord(\alpha) \notin D \setminus \{q-1\}$$

        \item Yes, $R$ contains zero divisors.
              From part (b), we know that $f(X) = X^4 + 2X^3 + X + 2$ is reducible in $\F_5[X]$, factoring as:
              $$f(X) = (X+1)(X+2)(X^2+4X+1)$$
              Since $f(X)$ is not irreducible, the quotient ring $R = \F_5[X]/\gen{f(X)}$ is not a field
              and therefore contains zero divisors (for example, the coset $(X+1) + \gen{f(X)}$),
              because $(X+1)$ is a proper monic factor of $f(X)$.

        \item \textit{NOTE: In this exercise, we use a different $f$, namely
                  $f(X)=X^3 + X-2$.}

              An element $g(X) + \gen{f(X)} \in R$ in this quotient group is either
              a zero-element, a unit, or a zero-divisor.
              Because we can write each coset in standard form,
              let's assume all these elements $g(X) + \gen{f(X)}$ are in standard form.
              \begin{itemize}
                  \item The only zero-element is $0 + \gen{f(X)} = \gen{f(X)}$.
                  \item Units are elements with $\deg(\gcd[f(X), g(X)]) = 0$.
                  \item Zero-divisors are elements with $0 < \deg(\gcd[f(X), g(X)]) < \deg(f(X))$.
              \end{itemize}

              We can thus calculate the number of units $N_u$ by \textbf{first counting the number of zero-divisors} $N_z$.
              We know that the total number of elements in $R$ is $|R| = p^d = 5^3 = 125$.
              The ring is partitioned into units, zero-divisors, and the zero element:
              $$|R| = N_u + N_z + 1$$

              To find $N_z$, we analyze the factorization of $f(X)$ in $\mathbb{Z}_5[X]$ to determine
              how many zero divisors there are.
              We find the following factorization into irreducible polynomials in $\mathbb{Z}_5[X]$:
              $$ X^3 + X + 3 = (X+4)(X^2 + X + 2) $$
              The quadratic factor $X^2 + X + 2$ is irreducible in $\mathbb{Z}_5$ because
              it has no roots (checking values $0, 1, 2, 3, 4$ yields no zeros). $(X+4)$ is
              a factor of degree $1$, which is always irreducible.

              Above, we found that a non-zero coset $g(X) + \gen{f(X)}$ in standard form
              (with thus $g(X) \neq 0$ and $\deg(g(X)) < \deg(f(X))$) is a zero-divisor if and only if $\deg(\gcd[g(X), f(X)]) > 0$.
              Furthermore, it must be a multiple of a proper (monic) factor of $f(X)$.

              Thus, for a zero-divisor $g(X) + \gen{f(X)}$, $\gcd[f(X), g(X)]$ is either $X+4$
              or $X^2+X+2$, because the GCD is \textit{unique} if it's monic.
              We count the possibilities for $g(X)$ in these two disjoint cases:

              \begin{enumerate}
                  \item \textbf{Case 1: $\boxed{\gcd[f(X), g(X)] = X+4}$}

                        The polynomial $g(X)$ must be a multiple of $X+4$.
                        Further, since it is in standard form, $\deg(g(X)) < 3$.
                        From these two conditions, we find that $g(X)$ must take the form:
                        $$ g(X) = (X+4)(aX+b) $$
                        where $a, b \in \mathbb{Z}_5$. There are $5 \cdot 5 = 25$ choices for the pair $(a,b)$.
                        Excluding the zero-coset case where $(a,b)=(0,0)$ (giving $g(X)=0$), we have:
                        $$ 25 - 1 = 24 \text{ zero-divisors.} $$

                  \item \textbf{Case 2: $\boxed{\gcd[f(X), g(X)] = X^2+X+2}$}

                        The polynomial $g(X)$ must be a multiple of $X^2+X+2$.
                        Further, since it is in standard form, $\deg(g(X)) < 3$.
                        From these two conditions, we find that $g(X)$ must take the form:
                        $$ g(X) = a(X^2 + X + 2) $$
                        where $a \in \mathbb{Z}_5$. There are 5 choices for $a$. Excluding $a=0$, we have:
                        $$ 5 - 1 = 4 \text{ zero-divisors.} $$
              \end{enumerate}

              Summing these cases, the total number of zero-divisors is:
              $$ N_z = 24 + 4 = 28 $$

              Finally, we calculate the number of units $N_u$.
              Recalling that the total number of elements is $|R| = p^d = 5^3 = 125$:
              $$ N_u = |R| - 1 - N_z = 125 - 1 - 28 = 96 $$

              Thus, $R$ contains \textbf{96 units}.

        \item Let's find the multiplicative order of $\alpha$, i.e. the integer
              $i > 0$ such that $\alpha^i = 1_R = 1 + \gen{f(X)}$.

              Note that $\deg(\alpha) < \deg(f(X))$, in other words, it is \textbf{already in standard form}.
              This means that $\alpha \neq 0 + \gen{f(X)}$. Furthermore, this also means
              that $\alpha \neq 1 + \gen{f(X)}$ and thus $\ord(\alpha) \neq 1$.

              We compute $\alpha^2$:
              $$ \alpha^2 = [X + \gen{f(X)}] \cdot [X + \gen{f(X)}] = X^2 + \gen{f(X)} $$
              This is not in standard form, since $\deg(X^2) = 2 \not< \deg(f(X))$, so we
              perform long division of $\red{X^2 \text{ by } f(X) \text{!}}$ to find the standard form
              $$ \alpha^2 = \red{-}(X+1) + \gen{f(X)} \neq 1 + \gen{f(X)}$$
              For $\alpha^3$, note that $\alpha^3 = \alpha \cdot \alpha^2 = -(X^2 + X) + \gen{f(X)} \red{= 1 + \gen{f(X)}}$.
    \end{enumerate}
\end{answer}

% \textbf{NOTE:}\\
% To find an irreducible polynomial of degree 5 in $\F_2[X]$, it is sufficient to find a polynomial of degree five that has no factors of degree 1 or 2:
% Why?
% \begin{itemize}
%     \item If $f(X)$ is reducible into factors $f(X) = g(X) \cdot h(X)$, then $(\deg(g(X)), \deg(h(X))) = (1,4)$ or $(2,3)$.
%     \item If $f(X)$ has 3 factors, then $(\deg(g(X)), \deg(h(X)), \deg(k(X))) = (1,1,3)$ or $(1,2,2)$.
%     \item If $f(X)$ has 4 factors, then $(\deg(g(X)), \deg(h(X)), \deg(k(X)), \deg(l(X))) = (1,1,1,2)$.
%     \item If $f(X)$ has 5 factors, then $(\deg(g(X)), \deg(h(X)), \deg(k(X)), \deg(l(X)), \deg(m(X))) = (1,1,1,1,1)$.
% \end{itemize}
% Finding factors of degree one is equivalent to finding roots.
% The only irreducible factors of degree 1 are $X$ and $X + 1$, while a direct
% computation shows that the only irreducible polynomial in $\F_2[X]$ of
% degree two equals $X^2 + X + 1$.

\end{document}
